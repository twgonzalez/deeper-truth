\documentclass{article}
\usepackage{amsmath, amssymb, amsfonts}
\usepackage{graphicx}
\usepackage{hyperref}
\usepackage{physics}

\title{SRFT Fractional PDE Expansion Plan}
\author{}
\date{}

\begin{document}

\maketitle

\section{Phase 0: Baseline (Existing MVP)}

\textbf{Goal:} Establish a stable proof-of-concept PDE that demonstrates amplitude lumps, threshold switching, and memory effects.

\subsection{Theoretical}
Finalize well-posedness for the single-domain or extended-manifold PDE, analyzing blowup behavior and amplitude triggers.
Ensure a simple yet stable simulation that captures lumps forming or saturating in 1D or 2D.

\subsection{Numerical / Toy Simulation}
Implement a basic fractional PDE solver:
\begin{equation}
    \partial_t^{\alpha} A + (-\Delta)^s A = F(A)
\end{equation}
in 1D.
Reproduce threshold-based blowups or stable lumps.
Verify iterative convergence and energy bounds.

\textbf{Outcome:} A "hello world" implementation of the SRFT PDE as proof-of-concept.

\section{Phase 1: Introduce a Simple Newtonian Potential Coupling}

\textbf{Motivation:} Emulate Newtonian gravity by introducing a mass-like amplitude-dependent potential.

\subsection{Theoretical}
Introduce a second variable $\Phi(x)$ governed by a Poisson equation:
\begin{equation}
    L_z^s g \Phi = \kappa |A|^2, \quad \text{or} \quad \Delta \Phi = C |A|^2.
\end{equation}
Couple $\Phi$ back into the PDE for $A$ with a forcing term:
\begin{equation}
    - \nabla \Phi \cdot \nabla A.
\end{equation}
Prove existence/uniqueness in a low-dimensional setting.

\subsection{Numerical / Toy Simulation}
Solve a coupled PDE system iteratively:
\begin{enumerate}
    \item Solve fractional PDE for $A$.
    \item Solve $\Delta \Phi = |A|^2$.
    \item Update $A$ with $\Phi$ coupling.
\end{enumerate}
Observe if lumps "clump together" under the potential.

\textbf{Outcome:} A minimal model showing lump-induced gravitational attraction.

\section{Phase 2: Expand Domain and Add Electromagnetism}

\textbf{Motivation:} Introduce a $U(1)$ gauge field for electromagnetism.

\subsection{Theoretical}
Introduce field variables:
\begin{equation}
    A(x), \quad A_\mu(x), \quad \text{or} \quad F_{\mu \nu}.
\end{equation}
Couple Maxwell's equations to amplitude lumps:
\begin{equation}
    \rho = |A|^2.
\end{equation}

\subsection{Numerical / Toy Simulation}
Discretize and solve:
\begin{enumerate}
    \item Fractional wave equation for lumps.
    \item Maxwell equations with charge density $\rho$.
    \item Observe charge-dependent lump interactions.
\end{enumerate}

\textbf{Outcome:} A toy "SRFT + U(1)" model demonstrating charge-carrying lumps.

\section{Phase 3: Merge Newtonian Gravity and U(1) in 2D}

\textbf{Motivation:} Investigate interactions between gravity-like and electromagnetic fields in a toy PDE system.

\subsection{Theoretical}
Combine the PDE system from Phases 1 and 2.
Attempt a partial existence proof for a fractional PDE system in 2D.

\subsection{Numerical / Toy Simulation}
Solve iteratively in a 2D domain:
\begin{enumerate}
    \item Update amplitude lumps.
    \item Solve $\Phi$ for gravity.
    \item Solve $A_\mu$ for EM.
    \item Observe combined interactions.
\end{enumerate}

\textbf{Outcome:} A system where lumps generate gravitational wells and electromagnetic repulsion.

\section{Phase 4: Extend to Weak Force (SU(2) Gauge Group)}

\textbf{Motivation:} Introduce weak isospin states.

\subsection{Theoretical}
Extend the PDE to allow a two-component amplitude:
\begin{equation}
    \Psi = (\psi_1, \psi_2).
\end{equation}
Couple to an $SU(2)$ gauge potential $W_\mu^a$.

\subsection{Numerical / Toy Simulation}
Implement a 1D/2D solver including the weak force gauge field.
Observe how doublets transform and interact.

\textbf{Outcome:} A toy weak-force model.

\section{Phase 5: Preliminary Strong Force (SU(3) Gauge Group)}

\textbf{Motivation:} Explore color charge interactions.

\subsection{Theoretical}
Introduce color-charged lumps in an $SU(3)$ representation.
Analyze potential confinement effects.

\subsection{Numerical / Toy Simulation}
Simulate a 1D or small-lattice approach to check for color singlet formation.

\textbf{Outcome:} A minimal QCD-like PDE with color confinement properties.

\section{Phase 6: Emergent Metric for Gravity}

\textbf{Motivation:} Move from Newtonian to metric-based gravity.

\subsection{Theoretical}
Define stress-energy tensor $T_{\mu\nu}$ and link curvature to lumps:
\begin{equation}
    R_{\mu\nu} = \kappa T_{\mu\nu}.
\end{equation}

\subsection{Numerical / Toy Simulation}
Simulate a toy 2D "mini-spacetime" solving Einstein-like equations dynamically.

\textbf{Outcome:} A first attempt at emergent curvature from lump interactions.

\section{Phase 7: Unifying All into a GUT-Like PDE}

\textbf{Motivation:} Integrate EM, Weak, Strong, and Gravity into a single PDE framework.

\subsection{Theoretical}
Develop a unified PDE system incorporating:
\begin{equation}
    \Psi, (A_\mu, W_\mu, G_\mu), g_{\mu\nu}.
\end{equation}
Analyze consistency and feasibility.

\subsection{Numerical / Toy Simulation}
Develop a high-performance computing (HPC) implementation to simulate full interactions.

\textbf{Outcome:} A preliminary grand unified PDE model.

\section{General Methodological Guidelines}

\begin{itemize}
    \item Incrementally add physics layers to ensure stability.
    \item Prioritize local or short-time existence proofs.
    \item Focus on 1D/2D simulations to manage computational complexity.
    \item Validate against simpler known PDE models.
    \item Maintain clear documentation of incremental findings.
\end{itemize}

\textbf{Conclusion:} This stepwise approach allows for incremental breakthroughs in understanding fractional PDEs and their connection to gauge fields and emergent geometry.

\end{document}
