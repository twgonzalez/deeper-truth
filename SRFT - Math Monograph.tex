\documentclass[12pt]{article}

\usepackage{amsmath,amssymb,amsthm,amsfonts,geometry,hyperref}
\usepackage{mathtools}  % for \coloneqq
\geometry{margin=1in}

\newtheorem{theorem}{Theorem}[section]
\newtheorem{lemma}[theorem]{Lemma}
\newtheorem{proposition}[theorem]{Proposition}
\newtheorem{remark}[theorem]{Remark}
\newtheorem{definition}[theorem]{Definition}
\newtheorem{corollary}[theorem]{Corollary}


\title{\Large \textbf{A Unified Fractional PDE Framework for Self Referential Field Theory: A Monograph}\\[6pt]
}
\author{%
  Thomas Gonzalez \\
  \small \texttt{twgonzalez@gmail.com}
}
\date{\today}

\begin{document}

\maketitle

\vfill
\noindent\hrulefill
\medskip

\noindent 
This work is licensed under the MIT License. You may use, copy, modify, and distribute this work provided that you include the original copyright and license notice.  For more details, see the full license text below or visit: \texttt{https://opensource.org/licenses/MIT}.
\medskip \\
\rightline{\textbf{\textcopyright~2025 Thomas Gonzalez}}
\newpage
\noindent
\begin{abstract}
This monograph develops a unified framework for \emph{fractional partial differential equations} (PDEs) that seamlessly integrates \textbf{memory effects}, \textbf{amplitude-triggered threshold logic}, and \textbf{wave interference} into a single advanced mathematical setting. We begin by establishing classical well-posedness results for Caputo time-fractional PDEs with constant or variable orders in a single bounded domain. Using a Galerkin projection approach and fractional Gr\"onwall inequalities, we establish rigorous existence, uniqueness, and regularity for both Lipschitz and monotone forcing. Although time is introduced as an external evolution parameter to facilitate mathematical analysis, our framework supports the metaphysical view that time is \emph{emergent}—arising from the recursive, self-referential dynamics of the underlying system rather than being fundamental.

Next, we incorporate \textbf{amplitude-dependent fractional exponents}: the PDE switches to higher-order fractional memory regimes once the solution amplitude exceeds a critical threshold, thus enabling finite-time blowups, lumps, or saturations under carefully chosen forcing laws. We address \emph{local vs.\ global memory reinitialization} (see Remark~\ref{rem:memory_reinit}) and demonstrate how blowup scenarios are treated via piecewise PDE definitions, preserving continuity of solutions at threshold times.

Finally, we extend the PDE to a \emph{higher-dimensional manifold} \( \mathcal{M} = \Omega \times \mathcal{A}\), unifying \textbf{wave interference} in physical coordinates with \textbf{amplitude-triggered thresholds} in an augmented amplitude dimension. Through harmonic forcing and fractional operators in both space and amplitude, we capture phenomena reminiscent of wave-particle duality: repeated PDE evolutions generate detection events localized by interference and governed by amplitude thresholds. A probability-like density emerges by integrating out the amplitude coordinates, tying wave phenomena to discrete “clicks” in a purely deterministic PDE framework.

Throughout, we provide detailed proofs, threshold analyses, and memory reinitialization strategies, forming a coherent, self-contained treatment of fractional PDEs with amplitude-based switching and wave interference. Our results open new avenues for modeling fractal-like or quantum-inspired processes in a unified mathematical setting.
\end{abstract}
\end{abstract}

\newpage

\tableofcontents

\newpage

%=========================================================
\section*{Introduction}
\addcontentsline{toc}{section}{Introduction}
\label{sec:intro_monograph}
%=========================================================

Fractional partial differential equations (PDEs) have received increasing attention in recent 
years for modeling \emph{nonlocal memory} and \emph{anomalous diffusion} in physics, 
biology, and engineering \cite{Diethelm2010,Kilbas2006}. Unlike classical PDEs, 
\textbf{fractional derivatives} capture long-range temporal or spatial correlations, 
leading to more accurate representations of subdiffusive processes, heavy-tailed jumps, 
and fractal-like dynamics. While many foundational results exist for fixed-order 
fractional PDEs in a single domain, real-world phenomena often exhibit 
\emph{variable-order memory}, \emph{amplitude-triggered blowups}, and even 
\emph{wave-particle} interplay—all of which require more flexible PDE frameworks.

\paragraph{Motivation and Scope.}
The overarching goal of this monograph is to develop a \emph{unified approach} to 
fractional PDEs encompassing:
\begin{itemize}
  \item \textbf{Variable or amplitude-based fractional orders}, allowing 
        \(\partial_t^\alpha\) and \((- \Delta)^s\) to \emph{switch} when the solution’s 
        amplitude crosses a threshold,
  \item \textbf{Wave or harmonic forcing}, inducing interference patterns in a 
        \emph{physical} coordinate,
  \item \textbf{Higher-dimensional expansions}, permitting amplitude or probability 
        dimensions to coexist with wave interference in a single PDE,
  \item \textbf{Well-posedness under blowups or saturations}, supporting finite-time 
        amplitude spikes or lumps via amplitude-triggered transitions.
\end{itemize}
Our perspective, often called the \emph{Self Referential Field Theory (SRFT) viewpoint}, 
treats amplitude thresholds and wave interference as \emph{part} of the same PDE 
architecture, thus bridging continuous wave phenomena and discrete amplitude events.

\paragraph{Structure of the Monograph.}
Following this Introduction, a dedicated section on \textbf{Applications and Implications} (Section~\ref{sec:applications}) outlines the broader impact of our work—including potential applications in quantum-classical transitions, fractal systems, computational physics, and more. We then proceed with the technical development, which is divided into four main parts—\textbf{A--D}—each building on the previous:

\begin{enumerate}
\item \textbf{Part~A: Fixed-Order Fractional PDE in a Single Domain.}
  We begin by recalling the classical \emph{Caputo} fractional derivative $\partial_t^\alpha$ 
  with $0<\alpha<1$, coupled to a spectral fractional Laplacian $(-\Delta)^s$ (for 
  $0<s<1$) on a bounded domain $\Omega\subset\mathbb{R}^n$. We prove well-posedness 
  (existence, uniqueness, and regularity) via:
  \begin{itemize}
    \item \textbf{Galerkin approximation} in the eigenbasis of $(-\Delta)^s$,
    \item \textbf{Fractional Grönwall inequalities} controlling energy in $H_0^s(\Omega)$,
    \item \textbf{Aubin--Lions--Simon compactness} to pass to the limit as $N\to\infty$.
  \end{itemize}
  This sets the foundation for more complex scenarios.

\item \textbf{Part~B: Variable-Order Extensions.}
  Next, we allow the fractional exponent $\alpha(t)$ (and possibly $s(t)$) to 
  \emph{change in time}, reflecting evolving memory depth or fractality. 
  By splitting the time interval at points where $\alpha$ is piecewise constant or 
  smoothly varying, we adapt the single-domain argument to produce well-posed solutions. 
  A \textbf{Volterra stability} argument replaces the classical fractional Grönwall lemma, 
  handling continuously or piecewise varying exponents.

\item \textbf{Part~C: SRFT in a Single Domain—Amplitude-Triggered Blowups.}
  Many real systems require \emph{amplitude thresholds}: once $|A|$ exceeds 
  $A_{\mathrm{crit}}$, the PDE \emph{switches} fractional exponents or forcing, thus 
  allowing finite-time blowups, “lumps,” or saturations. We show how to:
  \begin{itemize}
    \item \textbf{Reinitialize or continue memory} at the threshold time $t_b$,
    \item Solve piecewise PDEs in \([0,t_b]\) and \([t_b,T]\), matching continuity at $t_b$,
    \item Still retain a unique solution or a maximal-time blowup scenario under 
          monotone or Lipschitz forcing $\mathcal{F}$.
  \end{itemize}
  This single-domain amplitude-triggered regime sets the stage for multi-dimensional 
  amplitude expansions.

\item \textbf{Part~D: Extended Manifold \(\Omega\times \mathcal{A}\).}
  Finally, we embed the PDE into a \emph{higher-dimensional} manifold 
  $\mathcal{M}=\Omega\times\mathcal{A}$, where $x\in\Omega$ (physical domain) and 
  $a\in\mathcal{A}$ (amplitude or probability dimension). This single PDE can then 
  capture:
  \begin{itemize}
    \item \textbf{Wave Interference} in $x$, e.g.\ with $\sin(\kappa\cdot x-\omega t)$,
    \item \textbf{Amplitude Thresholds} in $a$, enabling local blowups or lumps,
    \item \textbf{Probability-Like Interpretation} by integrating out $a$ to get 
          $\rho(t,x)=\int|U|^2\,da$, reminiscent of a Born-rule detection probability.
  \end{itemize}
  The same \emph{Galerkin + fractional Grönwall} approach applies, now with 
  eigenfunctions $\{\Phi_k\}$ in $\mathcal{M}$. We also discuss wave-particle analogies 
  and stable attractors across dimensions.
\end{enumerate}

\paragraph{Highlights and Contributions.}
The monograph’s \emph{main contributions} are:
\begin{itemize}
    \item A \textbf{unified well-posedness} result for fractional PDEs \emph{with or 
          without} threshold switching, both in single domains and extended manifolds;
    \item A \textbf{piecewise PDE strategy} to handle amplitude-based blowups or saturations, 
          including memory reinitialization at threshold times if desired;
    \item A \textbf{multi-dimensional PDE} on $\Omega\times\mathcal{A}$ that naturally 
          links \textbf{wave interference} with \textbf{amplitude triggers}, providing a 
          deterministic route to wave-particle-like phenomena in a single PDE framework;
    \item Detailed \textbf{fractional Grönwall, product rule, and Volterra stability} 
          techniques, adapted from classical works \cite{Diethelm2010,Kilbas2006} to 
          amplitude-triggered or variable-order contexts.
\end{itemize}

\paragraph{Overview of the Proof Techniques.}
Our approach consistently employs:
\begin{enumerate}
\item \textbf{Caputo ODE Projection:} Using an eigenbasis of $(-\Delta)^s$ or 
      $(-\Delta_{\mathbf{z}})^s$, we reduce the PDE to a system of Caputo-type ODEs for 
      the coefficient vector. 
\item \textbf{Fractional Energy Estimate:} Multiplying each ODE by $\lambda_j^s\,u_j$ 
      and summing yields an inequality 
      $\partial_t^\alpha E(t)\le a\,E(t)+b$, leading to uniform bounds via a fractional 
      Grönwall lemma.
\item \textbf{Time Splitting at Thresholds:} If amplitude triggers occur, we restrict 
      each PDE solution to intervals where the exponent (and forcing) remain in the 
      same “low” or “high” regime, ensuring continuity at crossing times.
\item \textbf{Aubin--Lions--Simon Compactness:} Boundedness in $H_0^s$ plus fractional 
      Volterra arguments yield convergence $U_N\to U$, verifying that $U$ solves the PDE 
      in a weak sense.
\item \textbf{Uniqueness via Difference‐of‐Solutions:} Standard fractional Grönwall or 
      monotonicity arguments ensure that if $\mathcal{F}$ is Lipschitz or monotone, 
      two solutions must coincide.
\end{enumerate}

\paragraph{Applications and Broader Significance.}
This framework paves the way for modeling:
\begin{itemize}
  \item \textbf{Phase Transitions or Blowups} in subdiffusive media, where amplitude 
        lumps can form and saturate,
  \item \textbf{Wave Interference with Nonlocal Memory}, linking classical wave fringes 
        to discrete amplitude thresholds,
  \item \textbf{Quantum-Inspired Phenomena}, where repeated PDE runs produce detection 
        events distributed by $\lvert U\rvert^2$ in $(x,a)$-space.
\end{itemize}

\paragraph{Reading Guide.}
Readers seeking a \emph{foundation in fractional PDEs} may focus on Part~A, which includes 
the classical Caputo derivative approach and standard well-posedness arguments. Those 
interested in \emph{variable-order memory} can proceed to Part~B. 
Amplitude-trigger enthusiasts or blowup specialists might turn to Part~C. 
Finally, the \emph{multi-dimensional wave-amplitude} interplay is fully presented in 
Part~D, completing the SRFT viewpoint.

\medskip

\noindent
We hope this monograph provides a clear path to unifying fractional memory, amplitude 
threshold switching, and wave interference in a single PDE, offering new insights 
into how seemingly discrete and continuous phenomena can emerge from one deterministic 
framework.


\medskip

\subsection*{Assumptions and Hypotheses}
\addcontentsline{toc}{section}{Assumptions and Hypotheses}
\label{sec:hypotheses}

In this monograph, our analysis is conducted under the following assumptions and conditions:

\begin{enumerate}
  \item \textbf{Domain Regularity:}  
    The physical domain $\Omega\subset\mathbb{R}^n$ is assumed to be bounded with a Lipschitz (or $C^{1,\alpha}$) boundary. In the extended setting (Part~D), the manifold is given by 
    \[
      \mathcal{M} = \Omega \times \mathcal{A},
    \]
    where $\mathcal{A}\subset \mathbb{R}^m$ is either a bounded set or equipped with appropriate decay conditions.
  
  \item \textbf{Fractional Exponents:}  
    \begin{itemize}
      \item For the fixed-order analysis (Part~A), the time-fractional order satisfies $0 < \alpha < 1$, and the spatial fractional exponent satisfies $0 < s < 1$.
      \item In the variable-order setting (Part~B), $\alpha(t)$ is continuous with
            \[
              0 < \alpha_{\min} \le \alpha(t) \le \alpha_{\max} < 1 \quad \text{for } t \in [0,T].
            \]
    \end{itemize}
  
  \item \textbf{Forcing TSRFT Regularity:}  
    The nonlinear forcing function $F:\mathbb{R}\to\mathbb{R}$ (or $\mathcal{F}$ in the extended model) is assumed to be globally (or one-sided) Lipschitz, ensuring that the energy estimates and uniqueness results hold.
  
  \item \textbf{Initial Data:}  
    The initial condition is given by $A_0\in H_0^s(\Omega)$ (or $U_0\in H_0^s(\mathcal{M})$) and is compatible with the imposed boundary or decay conditions.
  
  \item \textbf{Amplitude Thresholding (Parts~C and D):}  
    A critical amplitude $A_{\mathrm{crit}}$ is prescribed. When the solution amplitude exceeds $A_{\mathrm{crit}}$, the fractional exponents and/or the forcing tSRFT switch between the “low” and “high” regimes. In these cases, the Caputo derivative may either maintain global memory (integrating from time $0$) or have its memory reinitialized at the threshold time.
\end{enumerate}

\bigskip


\subsection*{Refined Notation and Terminology}
\label{subsec:notation_table}
%=========================
To maintain clarity across the four parts, we summarize the main notational conventions for
our solutions, fractional exponents, and the meaning of “amplitude.” 

\begin{itemize}
\item \textbf{Parts~A--C (Single-Domain):} 
  We use $A(t,x)$ as the solution, defined on $[0,T]\times\Omega$. 
  The fractional exponents are denoted by $\alpha(t)$ and $s(t)$ (possibly fixed or variable), 
  and the tSRFT “amplitude” refers to the \emph{magnitude} $\lvert A(t,x)\rvert$.

\item \textbf{Part D (Extended Domain):} 
  We switch to $U(t,\mathbf{z})$ where $\mathbf{z}=(x,a)\in \mathcal{M}=\Omega\times\mathcal{A}$. 
  Here, $a$ is treated as an \emph{independent amplitude coordinate}. If exponents vary across 
  $\mathcal{M}$, we write $\alpha(\mathbf{z})$, $s(\mathbf{z})$, etc. 
  In this setting, “amplitude” can also be the coordinate $a$ itself, so threshold switching 
  occurs if $\lvert U\rvert$ crosses a critical level \emph{within} the extended domain.

\item \textbf{Memory Conventions:} 
  Throughout, $\partial_t^\alpha$ denotes the \emph{Caputo} fractional derivative. We highlight 
  whether memory is \emph{global} or \emph{reinitialized} upon threshold crossings 
  (see Remark~\ref{rem:memory_reinit}).

\item \textbf{Blowups and Lumps:} 
  In Parts~C--D, amplitude-triggered switching can cause finite-time blowups or saturations 
  (\emph{lumps}). We treat these via piecewise definitions of $\alpha$, $s$, and $F$ 
  on subintervals where $\lvert A\rvert$ or $\lvert U\rvert$ remains below or above a 
  critical threshold.
\end{itemize}

This table and the above clarifications ensure that the change from $A(t,x)$ to $U(t,\mathbf{z})$
and from “amplitude $|A|$” to “amplitude coordinate $a$” will not cause confusion.

\section{Applications and Implications}
\label{sec:applications}
%=========================================================

\subsection{Quantum-Classical Transition}
SRFT offers a deterministic PDE mechanism for transitioning from smooth wave interference to discrete detection “clicks”:
\begin{itemize}
    \item \emph{Wave Interference in $\Omega$} shapes where amplitude grows large.
    \item \emph{Threshold Switching in $\mathcal{A}$} localizes lumps upon exceeding $A_{\mathrm{crit}}$.
\end{itemize}

\subsection{Fractals and Scale-Invariant Systems}
Amplitude-based feedback naturally generates fractal geometries and multi-scale blowups, applicable to:
\begin{itemize}
    \item Complex branching in biological systems,
    \item Large-scale structures in cosmology,
    \item Self-similar turbulence in fluid dynamics.
\end{itemize}

\subsection{Computational Physics and Beyond}
Embedding amplitude coordinates in $\mathcal{A}$ suggests new approaches for:
\begin{itemize}
    \item Valence shell transitions in atomic physics,
    \item Chaotic excitations in neural or ecological networks,
    \item Dimensional bridging between classical PDEs and quantum-like models.
\end{itemize}

\subsubsection{Bidirectional Cross-Scale Influences}
Within the SRFT framework, cross-scale influences are inherently bidirectional. In our model, high-dimensional amplitude waves not only trigger local threshold events in lower-dimensional projections, but discrete, stable structures (or “lumps”) that emerge at a lower scale also feed back to modulate and reshape the global wave state at higher dimensions. This bidirectional coupling is relevant for several reasons:
\begin{itemize}
    \item \textbf{Emergent Dynamics:} Feedback from lower-scale structures can reinforce or modulate global wave patterns, leading to self-organized behavior often observed in complex systems.
    \item \textbf{Multi-scale Synchronization:} Such cross-scale interactions facilitate synchronization between local and global dynamics, potentially explaining phenomena such as sudden phase transitions or shifts in the overall system behavior.
    \item \textbf{Unified Framework for Diverse Applications:} In areas ranging from quantum-classical transitions and neural network dynamics to turbulent flows and cosmology, the interplay between micro-scale (local) structures and macro-scale (global) fields is essential. A model that inherently incorporates bidirectional influences thus offers a unified approach to understanding these phenomena.
    \item \textbf{Path to Unifying Theories:} Capturing the reverse flow of influence provides insights into how discrete, localized events (such as quantum detection “clicks” or neural spikes) can collectively impact and reorganize the higher-dimensional state, potentially paving the way for new theoretical approaches to emergent behavior.
\end{itemize}
In summary, the bidirectional cross-scale coupling not only enriches the descriptive power of the SRFT framework but also opens promising avenues for exploring multi-scale feedback mechanisms in complex systems.


\subsection{Speculative Extensions to Spacetime \& Gravity}
One especially intriguing idea (albeit highly speculative at this stage) is to explore whether these cross-scale attractors could provide a basis for interpreting phenomena typically associated with spacetime curvature or gravitational “wells.” If future work demonstrates that the SRFT approach can unify wave states, memory operators, and lower-dimensional “mass-like” lumps in a self-consistent way, one might envision connections to emergent geometry, possibly linking to broader ideas in quantum gravity or fundamental constants such as $\hbar$. While this line of inquiry would require substantial theoretical development, it highlights the potential for SRFT to serve as a versatile framework extending into domains where both quantum and gravitational phenomena are relevant.

%=========================================================
\section{Preliminaries on Fractional Operators}
%=========================================================

%---------------------------------------------------------------------------
\subsection{Caputo Fractional Derivative}
\label{subsec:caputo}
%---------------------------------------------------------------------------

We begin by recalling the \emph{Caputo} definition of the fractional derivative for
\(\alpha\in(0,1)\). Let \(u \colon [0,T]\to \mathbb{R}\) be a function that is at least
continuously differentiable on \([0,T]\). Then the \(\alpha\)-order \textbf{Caputo
fractional derivative} of \(u\) at \(t\) is defined by
\begin{equation}
\label{eq:caputo_def_expanded}
\partial_t^\alpha u(t)
\;=\;
\frac{1}{\Gamma\bigl(1-\alpha\bigr)}
\int_{0}^{t}
  \frac{u'(\tau)}{\bigl(t-\tau\bigr)^{\alpha}}
\,d\tau,
\quad
\alpha\in(0,1).
\end{equation}
Here \(\Gamma(\cdot)\) is the usual Gamma function. For standard proofs of well-definedness
and basic properties (linearity, integration by parts, etc.), see
\cite[Chapter~2]{Diethelm2010} or \cite[Section~2.4]{Kilbas2006}.

\begin{remark}[Interpretation and Regularity Requirements]
The Caputo derivative \(\partial_t^\alpha u(t)\) can be viewed as a \emph{convolution} of
the ordinary derivative \(u'(\tau)\) against the kernel \((t-\tau)^{-\alpha}\). Consequently,
\(u\) must be at least \(C^1([0,T])\) (or piecewise \(C^1\) if we allow amplitude-triggered
switching in subintervals), so that the integral in \eqref{eq:caputo_def_expanded} converges
near \(\tau=t\). One also needs \(u'\) to be suitably continuous near \(\tau=0\) for the
Volterra integral to behave well. Various fractional PDE frameworks relax these smoothness
conditions, but classical arguments typically assume \(u\in C^1([0,T])\).
\end{remark}

\paragraph{Variable-Order \(\alpha(t)\).}
In some problems, \(\alpha\) depends on \(t\), or more generally on \((t,x)\) or \((t,A)\).
The definition then modifies to
\[
\partial_t^{\alpha(t)} u(t)
\;=\;
\frac{1}{\Gamma\bigl(1-\alpha(t)\bigr)}
\int_0^t
  \frac{u'(\tau)}{\bigl(t-\tau\bigr)^{\alpha(t)}}
\,d\tau,
\]
provided \(\alpha(t)\) stays within \((0,1)\) and satisfies certain continuity or boundedness
conditions (e.g.\ \(\alpha_{\min}\le\alpha(t)\le\alpha_{\max}<1\)). The integral is
interpreted piecewise in time if \(\alpha\) jumps at amplitude thresholds. For rigorous
treatment of variable-order definitions and continuous dependence on \(\alpha(t)\), see
\cite{Sun2019}.

\paragraph{A Brief Look at the Fractional Product Rule.}
A key tool for energy estimates in fractional PDEs is the so-called 
“fractional product rule,” which generalizes the classical integration by 
parts. In short, if $u,v$ are sufficiently smooth and $\alpha\in(0,1)$, 
then one often has
\[
\int_{0}^{t} u(\tau)\,\partial_\tau^\alpha v(\tau)\,d\tau
\;=\;
(-1)^m \!\int_{0}^{t} v(\tau)\,\partial_\tau^\alpha u(\tau)\,d\tau
\;+\;
\text{(possible boundary term)}.
\]
When certain boundary or initial conditions hold (e.g.\ $v(0)=0$), that boundary 
tSRFT vanishes, giving a clean identity that underpins PDE energy estimates.

For a \emph{complete, rigorous statement} (including boundary terms, variable-order 
versions, and references), we refer the reader to 
\textbf{Lemma~\ref{lem:frac_product_rule}} in Section~\ref{sec:general_lemmas}. 
We also note that memory reinitialization at threshold times can affect whether 
the boundary terms vanish (see Remark~\ref{rem:memory_reinit}). 

\paragraph{Concluding Remarks for the Caputo Derivative.} 
\begin{itemize}
\item{Volterra Nature:} The Caputo operator \(\partial_t^\alpha\) is a Volterra integral
   operator, introducing \emph{long-range memory} for \(0<\alpha<1\). In PDEs, this
   necessitates piecewise or global time integration, especially if \(\alpha\) changes.

\item{Piecewise in Time:} When amplitude thresholds cause \(\alpha\) or \(s\) to change at
   \(t_b\), one treats \(\partial_t^\alpha\) (or \(\partial_t^{\alpha_{\mathrm{high}}}\))
   on each interval \([t_{i-1},t_i]\). Either the memory integral reinitializes or continues
   from 0, depending on the physical model (Remark~\ref{rem:memory_reinit}).

\item{Utility in PDEs:} The combination of the fractional product rule and the fractional
   Grönwall inequality (introduced in the next subsection) provides a powerful mechanism
   for deriving \emph{energy estimates} and proving well-posedness in fractional PDEs under
   amplitude-triggered or piecewise definitions of \(\alpha\).
\end{itemize}
Thus, we have laid the foundation for how the \(\alpha\)-order Caputo derivative will be
applied in our fractional PDE analysis, both in the single-domain and extended-domain
contexts.
%---------------------------------------------------------------------------
\subsection{Memory Reinitialization vs.\ Global Memory}
\label{rem:memory_reinit}
%---------------------------------------------------------------------------
In fractional models—especially those with \emph{variable-order} derivatives or 
\emph{amplitude-triggered} exponents—it is often necessary to clarify whether the 
Caputo integral kernel maintains \emph{global memory} from time $0$, or whether 
\emph{memory is reinitialized} each time the system enters a new regime (for instance, 
when $\alpha$ switches at a threshold time $t_b$).

\paragraph{Global Memory.}
Under a global-memory convention, if the fractional derivative is
\[
\partial_t^{\alpha(t)} \,u(t) 
\;=\;
\frac{1}{\Gamma\!\bigl(1-\alpha(t)\bigr)}
\int_{0}^{t}
  \frac{u'(\tau)}{(t-\tau)^{\alpha(t)}}
\,d\tau,
\]
then the lower limit of integration remains $0$ even if $\alpha(t)$ or other PDE coefficients 
jump at intermediate times. Physically, this means the system “remembers” all past states 
continuously, regardless of regime changes.

\paragraph{Reinitialized (Partial) Memory.}
Alternatively, one may \emph{reinitialize} the fractional integral at the latest switching 
time $t_b$. In that case, for $t>t_b$, 
\[
\partial_{t}^{\alpha_{\mathrm{high}}} u(t)
\;\;=\;\;
\frac{1}{\Gamma\!\bigl(1-\alpha_{\mathrm{high}}\bigr)}
\int_{t_b}^{t}
  \frac{u'(\tau)}{(t-\tau)^{\alpha_{\mathrm{high}}}}
\,d\tau,
\]
with new initial conditions $u(t_b^+)=u(t_b^-)$ ensuring continuity. This models the 
system “forgetting” pre‐switch data, effectively resetting the Caputo integral at $t_b$. 
Such a choice may arise in applications where crossing a threshold physically erases 
older memory, e.g.\ phase transitions, certain viscoelastic processes, or amplitude 
triggered neural resets.

\paragraph{Piecewise PDE Formulation.}
In both approaches, one typically divides $[0,T]$ into subintervals $[t_{i-1}, t_i]$ 
where $\alpha(t)$ (and other PDE parameters) stay within a single “regime.” 
- \textbf{Global memory} keeps using the integral from $0$ to $t$. 
- \textbf{Reinitialized memory} restarts the integral from the boundary of each subinterval.  
Either way, we solve a fractional PDE on each subinterval and match solutions (and 
memories) at the transition points. 

\paragraph{Choice Depends on the Model.}
The decision between global vs.\ partial memory is a modeling question; different 
physical, biological, or engineering contexts justify one or the other. Our PDE 
framework allows both conventions, and all main existence/uniqueness theorems 
remain valid once the memory convention is fixed and applied consistently.

%---------------------------------------------------------------------------
\subsection{Fractional Laplacian and Boundary Conditions}
\label{subsec:frac_laplacian}
%---------------------------------------------------------------------------

We next introduce the \emph{spectral fractional Laplacian} \((-\Delta)^s\), where 
\(s\in(0,1)\). Throughout, we let \(\Omega\subset\mathbb{R}^n\) be a bounded domain 
with sufficiently regular boundary \(\partial\Omega\) (e.g.\ Lipschitz or \(C^{1,\alpha}\)) 
to ensure the usual elliptic theory. We impose \emph{Dirichlet boundary conditions}
(\(u=0\) on \(\partial\Omega\)) in the sense described below.

\paragraph{Dirichlet Eigenbasis of the Laplacian.}
Consider the classical Dirichlet eigenvalue problem:
\[
-\Delta \,\phi_k(x) = \lambda_k \,\phi_k(x), 
\quad
\phi_k\big|_{\partial \Omega} = 0,
\quad
x\in \Omega.
\]
By standard elliptic PDE theory, one obtains a sequence of eigenfunctions
\(\{\phi_k\}_{k=1}^\infty\) and corresponding eigenvalues 
\(\{\lambda_k\}_{k=1}^\infty \subset (0,\infty)\), with 
\(\lambda_1 \le \lambda_2 \le \dots \to \infty\). The family 
\(\{\phi_k\}\) can be taken as an orthonormal basis of \(L^2(\Omega)\) 
(under the usual inner product), i.e.:
\[
\int_\Omega \phi_j(x)\,\phi_k(x)\,dx = \delta_{jk},
\]
and each \(\phi_k\in C^\infty(\Omega)\cap C^0(\overline{\Omega})\) (under suitable regularity).

\paragraph{Spectral Definition of \((- \Delta)^s\).}
Given \(s\in(0,1)\), we define \(\displaystyle (-\Delta)^s\) \emph{spectrally} as follows.
Any \(u\in L^2(\Omega)\) can be expanded in the eigenbasis:
\[
u(x)
\;=\;
\sum_{k=1}^\infty \hat{u}_k\,\phi_k(x),
\quad
\hat{u}_k
\;=\;
\int_\Omega u(x)\,\phi_k(x)\,dx.
\]
Then we \textbf{define}
\[
(-\Delta)^s u
\;=\;
\sum_{k=1}^\infty \lambda_k^s \,\hat{u}_k\,\phi_k(x).
\]
By construction, if \(\phi_k\) satisfies \(-\Delta \,\phi_k = \lambda_k\,\phi_k\), then
\[
(-\Delta)^s \phi_k(x)
\;=\;
\lambda_k^s\,\phi_k(x).
\]
Hence, \(\{\phi_k\}\) is also an eigenbasis for the fractional operator with eigenvalues
\(\{\lambda_k^s\}\).

\paragraph{Domain \(H_0^s(\Omega)\).}
We let \(H_0^s(\Omega)\) be the domain (or “energy space”) of \((- \Delta)^s\) under
Dirichlet boundary conditions.  Concretely, we say \(u\in H_0^s(\Omega)\) if
\[
u(x) 
= \sum_{k=1}^\infty \hat{u}_k\,\phi_k(x)
\quad\text{with}\quad
\sum_{k=1}^\infty \lambda_k^s\,\lvert \hat{u}_k\rvert^2 < \infty,
\quad
\text{and}\quad
u\big|_{\partial\Omega} = 0 \text{ (in the sense of traces).}
\]
Equivalently, one can equip \(H_0^s(\Omega)\) with the norm
\[
\|u\|_{H_0^s(\Omega)}^2
\;=\;
\sum_{k=1}^\infty
  \lambda_k^s \,|\hat{u}_k|^2,
\]
plus possibly an \(L^2(\Omega)\) tSRFT if one wants a fully equivalent norm.  
(Various authors define 
\(\|u\|_{H_0^s(\Omega)}^2 = \|(-\Delta)^{s/2} u\|_{L^2(\Omega)}^2\), 
but all definitions coincide up to constants \cite{DiNezzaPalatucciValdinoci}.)  

\begin{remark}[Integral Definition and Nonlocal Boundary Data]
An alternative viewpoint is the \emph{integral form} of the fractional Laplacian: 
\[
(-\Delta)^s u(x)
\;=\;
c_{n,s}
\;\mathrm{P.V.}
\int_{\Omega} 
  \frac{u(x) - u(y)}{\|x-y\|^{n+2s}}
\,dy,
\]
possibly extended outside \(\Omega\).  One imposes \(u=0\) on \(\partial\Omega\) by 
extending \(u\equiv 0\) in \(\Omega^c\). In a bounded domain with Dirichlet boundary, 
one has to be careful about the contribution from 
\(\{y\notin \Omega\colon \|x-y\|\) small\}. The \emph{spectral} definition used above is 
often simpler for PDE analysis, especially if we rely on the eigenfunction basis 
\(\{\phi_k\}\).  We refer to \cite{DiNezzaPalatucciValdinoci,Kilbas2006} for further 
comparisons.
\end{remark}

\paragraph{Boundary Condition Interpretation.}
When we say \(\phi_k\big|_{\partial\Omega}=0\), we mean in the usual classical or trace sense. 
For general \(u\in H_0^s(\Omega)\), the condition \(u|_{\partial\Omega}=0\) means that 
\(u\) belongs to the closure (in the fractional Sobolev norm) of 
\(\{v\in C_c^\infty(\Omega)\}\).  This ensures that the fractional operator \((- \Delta)^s\)
acts consistently with Dirichlet boundary data.  Physically, one can interpret 
\(\partial\Omega\) as forcing \(u=0\) or strongly damping outside \(\Omega\).

\paragraph{Using \((- \Delta)^s\) in PDEs.}
In fractional PDEs, an operator like 
\(\partial_t^\alpha u(t,x) + (-\Delta)^s u(t,x) = F(u)\) arises to model anomalous
diffusion or nonlocal boundary interactions.  The spectral expansion provides a 
\emph{natural basis} (\(\phi_k\)) for \emph{Galerkin approximation}, letting us 
truncate at mode \(N\) and approximate
\[
u_N(t,x)
= \sum_{k=1}^N a_k^N(t)\,\phi_k(x).
\]
Then \(\|u_N(t)\|_{H_0^s(\Omega)}^2 = \sum_{k=1}^N \lambda_k^s\,\lvert a_k^N(t)\rvert^2\).  
This approach is central in many well-posedness proofs.

---

Hence, by employing the **spectral fractional Laplacian** \((-\Delta)^s\) with Dirichlet 
boundary conditions, we gain a clear route to defining and approximating fractional PDE 
solutions via the \(\{\phi_k\}\) eigenbasis. For rigorous expositions of these expansions, 
one may consult \cite[Ch.~1]{Kilbas2006}, \cite[Sec.~3.1]{Diethelm2010}, or the extensive 
treatment in \cite{DiNezzaPalatucciValdinoci}.


\subsection{General Lemmas}
\label{sec:general_lemmas}
%=========================================================

In this section, we collect several standard results on fractional derivatives 
and Caputo ODEs that will be used throughout the monograph \cite{Diethelm2010, 
Kilbas2006, Ye2007, Sun2019}.

\begin{lemma}[Fractional Product Rule]
\label{lem:frac_product_rule}
Let $0 < \alpha < 1$ be fixed, and let $u,\,v$ be sufficiently smooth 
on $[0,T]$. Define the Caputo fractional derivative
\[
\partial_t^\alpha w(t)
\;=\;
\frac{1}{\Gamma(1-\alpha)}
\int_{0}^{t}
  \frac{w'(\tau)}{(t-\tau)^\alpha}\,d\tau.
\]
Then, under suitable boundary/initial conditions (for instance, $v(0)=0$), we have
\[
\int_{0}^{t} 
  u(\tau)\,\partial_\tau^\alpha v(\tau)\,d\tau
\;=\;
(-1)^m
\int_{0}^{t} 
  v(\tau)\,\partial_\tau^\alpha u(\tau)\,d\tau
\;+\;\text{(possible boundary term)},
\]
where $m$ depends on the precise variant (commonly $m=1$). 
If the boundary tSRFT vanishes (e.g.\ $v(0)=0$), the expression 
simplifies to
\[
\int_{0}^{t} u(\tau)\,\partial_\tau^\alpha v(\tau)\,d\tau
\;=\;
(-1)^m\!\int_{0}^{t} v(\tau)\,\partial_\tau^\alpha u(\tau)\,d\tau.
\]

\smallskip
\noindent
\emph{Variable \(\alpha(t)\).}
When $\alpha$ depends on $t$, a piecewise argument in subintervals 
$[t_{i-1}, t_i]$ where $\alpha$ is constant or Lipschitz typically applies. 
Boundary-type integrals at each sub-interval interface may arise; see 
\cite{Sun2019} for details.

\smallskip
\noindent
\textbf{Boundary Terms.}
Often called the “polarization” term, $\mathcal{B}[u,v]$, this boundary 
contribution typically vanishes under physically common conditions, like 
$v(0)=0$ or memory reinitialization at threshold times 
(see Remark~\ref{rem:memory_reinit}).
\end{lemma}

\begin{lemma}[Fractional Grönwall Inequality]
\label{lem:frac_gronwall}
There are two frequently used versions of the fractional Grönwall lemma:

\begin{enumerate}
\item[\textup{(a)}] \textbf{Constant \(\alpha\in (0,1)\).}
  Suppose \(y(t)\ge 0\) satisfies
  \[
    \partial_t^\alpha y(t)\;\le\; a\,y(t) + b,\quad t\in(0,T],\quad y(0)=y_0\ge0,
  \]
  for some constants \(a,b\ge0\). Then \(y(t)\) remains bounded on $[0,T]$ and in fact
  \[
    y(t)
    \;\le\;
    \Bigl(y_0 + \tfrac{b}{a}\Bigr)\,E_\alpha\bigl(a\,t^\alpha\bigr)
    \;-\;\tfrac{b}{a},
  \]
  where $E_\alpha(\cdot)$ is the Mittag–Leffler function.

\item[\textup{(b)}] \textbf{Variable‐order \(\alpha(t)\).}
  Let $\alpha(t)$ be a continuous function taking values in $[\alpha_{\min},\alpha_{\max}]\subset(0,1)$,
  and assume $y(t)\ge0$ satisfies
  \[
    \partial_t^{\alpha(t)}y(t) \;\le\; a\,y(t) + b,\quad y(0)=y_0\ge0.
  \]
  Then a similar bound (involving generalized Mittag–Leffler‐type functions) holds, ensuring
  $y(t)$ is uniformly bounded on $[0,T]$.  This follows from Volterra‐type stability arguments
  (see \cite{Sun2019,Ye2007}).
\end{enumerate}

In both cases, one obtains a uniform-in-time estimate that prevents blowups under linear‐type
forcing.  These bounds are crucial for energy estimates in fractional PDEs.
\end{lemma}

\begin{lemma}[Caputo ODE Existence and Uniqueness]
\label{lem:caputo_ODE_wellposed}
Consider the Caputo fractional ODE
\[
\partial_t^\alpha y(t)
\;=\;
f\bigl(t,y(t)\bigr),
\quad
0<\alpha<1,
\quad
t\in(0,T],
\quad
y(0)=y_0\in\mathbb{R}.
\]
Assume $f$ is globally Lipschitz in $y$.  Then there exists a unique continuous function
$y(t)$, $0\le t\le T$, solving this ODE in the Caputo sense.  Moreover, $y$ depends
continuously on the initial data $y_0$.  

If $f$ is only one‐sided Lipschitz or monotone (rather than globally Lipschitz), one
generally still obtains well‐posedness under mild conditions; see
\cite[Ch.~7]{Diethelm2010} and \cite[Sec.~2.3]{Kilbas2006} for a thorough discussion.
\end{lemma}

\medskip

\noindent
\textbf{Usage in the Monograph.}  
The above three lemmas underlie most of our subsequent proofs:
\begin{itemize}
\item \emph{Fractional product rule} for deriving energy estimates by multiplying
  PDEs/ODEs by the solution or its projection.
\item \emph{Fractional Grönwall} for closing those energy estimates and proving
  uniform boundedness.
\item \emph{Caputo ODE well‐posedness} for the Galerkin‐projection systems and
  piecewise PDE definitions in variable‐order or amplitude‐triggered regimes.
\end{itemize}
We will refer back to Lemmas \ref{lem:frac_product_rule}, \ref{lem:frac_gronwall},
and \ref{lem:caputo_ODE_wellposed} throughout Parts A--D.



%=========================================================
\section{Part A: Fixed‐Order Fractional PDE}
%=========================================================

In \textbf{Part~A}, we begin our analysis of \emph{time-fractional partial differential equations (PDEs)} in the simplest setting where the fractional order \(\alpha\in (0,1)\) is a \emph{fixed, constant} parameter. This scenario already captures a wide range of \emph{subdiffusive} or \emph{long-memory} phenomena, yet remains more tractable than the \emph{variable-order} regime considered later in Part~B.

\paragraph{Motivation.}
Many physical, biological, and engineering processes exhibit power-law memory kernels leading to a fractional derivative of a constant order \(\alpha\). Typical examples include subdiffusion in heterogeneous media, viscoelastic materials, and anomalous transport models. The \emph{Caputo} definition of the fractional derivative offers a convenient framework for imposing initial conditions consistently.

\paragraph{Key Objectives and Approach.}
\begin{itemize}
    \item \emph{Galerkin Approximation}: We will discretize the PDE in space using a finite number of Dirichlet eigenfunctions (the spectral fractional Laplacian approach). This produces a system of \emph{Caputo ODEs} with constant \(\alpha\). 
    \item \emph{Energy Estimates}: By employing the \textbf{fractional product rule} and \textbf{fractional Gr\"onwall} inequalities, we obtain a priori bounds on the Galerkin approximations. This ensures uniform boundedness in an \(H_0^s(\Omega)\) norm.
    \item \emph{Compactness and Passing to the Limit}: We invoke a \emph{fractional Aubin--Lions--Simon} argument (or “Volterra stability” viewpoint) to show that the approximate solutions converge to a true solution of the PDE. 
    \item \emph{Uniqueness under Lipschitz Forcing}: A fractional Gr\"onwall argument on the difference of two solutions ensures uniqueness if \(F\) is globally (or one-sided) Lipschitz. 
\end{itemize}

\paragraph{Content of Part~A.}
\begin{itemize}
    \item In the \textbf{Preliminaries and Notation} subsection, we define the spectral fractional Laplacian \((-\Delta)^s\) on \(\Omega\), recall the \emph{Caputo derivative} for a constant \(\alpha\in(0,1)\), and state core lemmas (fractional product rule and constant-\(\alpha\) Gr\"onwall inequality).
    \item Next, we \emph{formulate} the \textbf{model PDE}:
      \[
      \partial_t^\alpha A(t,x)
      \;+\;
      (-\Delta)^s A(t,x)
      \;=\;
      F\bigl(A(t,x)\bigr),
      \quad
      A(0)=A_0,\quad
      A\big|_{\partial\Omega}=0.
      \]
      We show how to build a \textbf{Galerkin approximation} \(A_N\) and derive an \emph{energy estimate} by multiplying the $j$th mode ODE by \(\lambda_j^s a_j^N\).
    \item We then demonstrate the \textbf{limit passage in $N$}: boundedness and fractional Volterra compactness allow us to extract a convergent subsequence $A_N \rightharpoonup A$, which in turn satisfies the PDE in the weak sense.
    \item Finally, we address \textbf{Uniqueness}: If \(F\) is globally Lipschitz, the difference of two solutions satisfies a fractional Gr\"onwall-type ODE, forcing them to coincide. 
\end{itemize}

\paragraph{Relation to Later Parts.}
In Part~B, we will extend these arguments to \emph{variable-order} \(\alpha(t)\). In Part~C, we will allow \(\alpha\) or \(s\) to switch upon amplitude thresholds, permitting blowups or multi-regime transitions. Nonetheless, the **core** ideas—Galerkin expansions, fractional Gr\"onwall bounds, Caputo ODE uniqueness—are already established here in \textbf{Part~A} for \(\alpha\) fixed.

\vspace{2ex}
\noindent
We now proceed to the \textbf{Preliminaries and Notation} for this fixed-order setting, laying out the fractional Laplacian domain, the Caputo operator, and the main lemmas that will guide our well-posedness analysis.
%---------------------------------------------------------------------------
\subsection{Preliminaries and Notation (Part A: Fixed-Order PDE)}
\label{subsec:prelim_partA}
%---------------------------------------------------------------------------

In this subsection, we gather the main definitions and notational conventions 
that underlie our analysis of the \textbf{fixed‐order time‐fractional PDE} on 
a bounded domain $\Omega$. We clarify the fractional Sobolev spaces for the 
spectral fractional Laplacian and the notion of the Caputo time derivative 
$\partial_t^\alpha$ with $\alpha \in (0,1)$.

\subsubsection{Domain, Fractional Laplacian, and Function Spaces}
\label{subsubsec:domain_spaces}

Let $\Omega\subset \mathbb{R}^n$ be a bounded domain with sufficiently regular 
boundary $\partial\Omega$ (for instance, Lipschitz or $C^{1,\alpha}$). We impose 
\emph{Dirichlet boundary conditions}, i.e.\ $u|_{\partial\Omega} = 0$ in the 
fractional sense. We denote by $(-\Delta)^s$ the \emph{spectral fractional 
Laplacian} of order $s\in(0,1)$, defined via the usual Dirichlet eigenfunctions 
$\{\phi_k\}$ and eigenvalues $\{\lambda_k\}$ of the (integer) Laplacian $-\Delta$:

\[
-\Delta \,\phi_k(x) 
= 
\lambda_k\,\phi_k(x),
\quad
\phi_k\big|_{\partial\Omega}=0,
\quad
\int_\Omega \phi_j(x)\,\phi_k(x)\,dx
=
\delta_{jk}.
\]

Then, for any $u\in L^2(\Omega)$ with the expansion 
$u(x) = \sum_{k=1}^{\infty} \hat{u}_k\,\phi_k(x)$, 
the \textbf{spectral fractional Laplacian} is given by
\[
(-\Delta)^s \,u
\;=\;
\sum_{k=1}^\infty
 \lambda_k^s\,\hat{u}_k\,\phi_k(x).
\]
We let $H_0^s(\Omega)$ denote the associated domain of $(-\Delta)^s$ with Dirichlet 
boundary conditions, equipped with the norm 
\[
\|u\|_{H_0^s(\Omega)}^2 
=\sum_{k=1}^{\infty} \lambda_k^s \,\lvert \hat{u}_k\rvert^2.
\]
This ensures $u(x)\big|_{\partial\Omega}=0$ in the fractional-trace sense; see 
\cite[Ch.~1]{Kilbas2006}, \cite[Sec.~3.1]{Diethelm2010}, or 
\cite{DiNezzaPalatucciValdinoci} for details.

\subsubsection{Caputo Fractional Derivative of Fixed Order \texorpdfstring{\(\alpha\in(0,1)\)}{}}
\label{subsubsec:caputo_fixed}

Throughout Part~A, we assume the time‐fractional order $\alpha$ is a 
\emph{fixed constant} in $(0,1)$. For a (suitably smooth) function 
$A:[0,T]\to \mathbb{R}$, the \emph{Caputo fractional derivative} of order 
$\alpha$ at time $t$ is

\begin{equation}
\label{eq:caputo_fixed_def}
\partial_t^\alpha A(t)
\;=\;
\frac{1}{\Gamma\bigl(1-\alpha\bigr)}
\int_0^t
  \frac{A'(\tau)}{(t-\tau)^\alpha}
\,d\tau,
\quad
0 < \alpha < 1.
\end{equation}

We refer to \cite[Ch.~2]{Diethelm2010} and \cite[Sec.~2.4]{Kilbas2006} for classical 
treatments of the Caputo operator, and to \cite[Ch.~7]{Diethelm2010} for how it appears 
in PDE or ODE existence proofs. Note that for $A\in C^1([0,T])$, the integral in 
\eqref{eq:caputo_fixed_def} converges near $t$; near $t=0$, we require $A'$ to be 
continuous enough so that $(t-\tau)^{-\alpha}$ is integrable.

For the fractional product rule and the fractional Grönwall inequality, we refer to Section~\ref{sec:general_lemmas}.


\subsubsection{Notation Summary}
\label{subsubsec:notation_summary}

For \textbf{Part~A (Fixed‐Order)}, we use:

\begin{itemize}
\item $\Omega\subset \mathbb{R}^n$: bounded domain; $0<s<1$ is the spatial 
      fractional exponent; $\alpha\in(0,1)$ is the **constant** time‐fractional order.
\item $(-\Delta)^s$: spectral fractional Laplacian with Dirichlet boundary; 
      $H_0^s(\Omega)$ the corresponding energy space, 
      $\|u\|_{H_0^s(\Omega)}^2 = \sum \lambda_k^s \,\lvert \hat{u}_k\rvert^2$.
\item $\partial_t^\alpha$: Caputo derivative of order $\alpha$, 
      see \eqref{eq:caputo_fixed_def}.
\item $F:\mathbb{R}\to\mathbb{R}$: often **globally Lipschitz** with constant $L_F$; 
      used as the nonlinear forcing in PDE.
\item $E_\alpha(\cdot)$: the Mittag‐Leffler function used in the fractional Gr\"onwall lemma.
\end{itemize}

With these definitions and lemmas in hand, the next subsections will analyze the 
\emph{time-fractional PDE of fixed order}:
\[
\partial_t^\alpha A(t,x)
\;+\;
(-\Delta)^s A(t,x)
\;=\;
F\bigl(A(t,x)\bigr),
\quad
A(0)=A_0,\quad
A\big|_{\partial\Omega}=0,
\]
constructing solutions via Galerkin approximations, obtaining uniform bounds from the 
fractional Gr\"onwall, and passing to the limit with a fractional Aubin--Lions--Simon 
argument for existence and uniqueness under Lipschitz $F$ and $A_0\in H_0^s(\Omega)$.




%---------------------------------------------------------------------------
\subsection{Model PDE and Assumptions}
\label{subsec:base_pde_expanded}
%---------------------------------------------------------------------------

We now focus on the following \emph{time-fractional diffusion-type} PDE on a bounded domain
\(\Omega\subset\mathbb{R}^n\):
\begin{equation}
\label{eq:fixedOrderPDE}
\begin{cases}
\displaystyle
\partial_t^\alpha A(t,x) 
\;+\; 
(-\Delta)^s A(t,x)
\;=\;
F\bigl(A(t,x)\bigr),
& t\in (0,T], \quad x\in \Omega,
\\[6pt]
A(0,x) = A_0(x), 
& x\in \Omega,
\\[4pt]
A(t,x) = 0,
& x\in \partial\Omega,\; t\in[0,T].
\end{cases}
\end{equation}
Here:
\begin{itemize}
\item \(\alpha\in(0,1)\) is the \textbf{fractional order in time}, and we interpret
  \(\partial_t^\alpha\) in the \emph{Caputo} sense (see
  Section~\ref{subsec:caputo}).
\item \(s\in(0,1)\) is the \textbf{fractional exponent in space}, and \((-\Delta)^s\)
  is the \emph{spectral} fractional Laplacian under Dirichlet boundary conditions
  (as introduced in Section~\ref{subsec:frac_laplacian}), ensuring \(A=0\) on
  \(\partial\Omega\).
\item The forcing (or reaction) tSRFT \(F:\mathbb{R}\to\mathbb{R}\) is \emph{globally
  Lipschitz} with a constant \(L_F\ge0\); that is, for all real numbers \(r_1,r_2\),
  \[
    \lvert F(r_1) - F(r_2)\rvert
    \;\le\; L_F\,\lvert r_1 - r_2\rvert.
  \]
\item The \textbf{initial data} \(A_0(x)\) lies in \(H_0^s(\Omega)\), the domain of
  \((-\Delta)^s\) with Dirichlet boundary. (This ensures we have a well-defined norm
  \(\|A_0\|_{H_0^s(\Omega)}<\infty\) and that \(A_0\) satisfies the boundary condition
  in the trace sense.)
\end{itemize}

\paragraph{Domain and Boundary Condition.}
We assume \(\Omega\subset\mathbb{R}^n\) is a bounded domain with sufficiently regular
boundary \(\partial\Omega\) (e.g.\ Lipschitz or \(C^{1,\alpha}\)). On \(\partial\Omega\),
we impose the Dirichlet-type condition
\[
  A(t,x) \;=\; 0,
  \qquad x\in \partial\Omega,\; t\in [0,T],
\]
which is understood in the spectral (or equivalently, fractional Sobolev) sense.  In
particular, if \(A(t,\cdot)\in H_0^s(\Omega)\) for each \(t\), that implies
\(A(t,\cdot)\) vanishes on \(\partial\Omega\) in the appropriate fractional-trace sense.

\paragraph{Fractional Derivative Interpretation.}
We interpret \(\partial_t^\alpha A(t,x)\) as the \emph{Caputo} fractional derivative of
order \(\alpha\).  Explicitly, for \(\alpha\in(0,1)\),
\[
\partial_t^\alpha A(t,x)
\;=\;
\frac{1}{\Gamma\bigl(1-\alpha\bigr)}
\int_{0}^{t}
\frac{\partial_\tau A(\tau,x)}{\bigl(t-\tau\bigr)^\alpha}\,d\tau.
\]
For well-posedness, we require \(\tau\mapsto A(\tau,x)\) to be at least piecewise
\(C^1\) in \(\tau\). We also assume \(A(0,\cdot)=A_0(\cdot)\) is compatible with the
boundary condition \(A_0|_{\partial\Omega}=0\) (which holds in the trace sense since
\(A_0\in H_0^s(\Omega)\)).

\paragraph{Assumptions on the Nonlinearity \(F\).}
We impose that \(F\colon \mathbb{R}\to\mathbb{R}\) is \textbf{globally Lipschitz}; i.e.,
for all real \(r_1,r_2\),
\[
\lvert F(r_1) - F(r_2)\rvert
\;\le\;
L_F\,\lvert r_1 - r_2\rvert,
\]
where \(L_F\) is a finite constant. This implies that \(F\) does not grow faster than linearly
and ensures certain \emph{energy estimates} (see Section~\ref{sec:energy_estimates}).  
Examples include \(F(r)=\lambda\,r\), a logistic-type function with bounded derivative, or
\(\tanh(r)\) on a bounded domain.  Under this Lipschitz condition, we will show that no
amplitude \emph{blowup} can occur for \eqref{eq:fixedOrderPDE} (in contrast to the
non-Lipschitz forcing case where blowups are possible).

\paragraph{Initial Condition in \(H_0^s(\Omega)\).}
We assume \(A_0(x)\in H_0^s(\Omega)\).  This requirement ensures that the initial data
satisfies the Dirichlet boundary condition (in the fractional-trace sense) and that we can
carry out Galerkin approximations in \(H_0^s(\Omega)\).  In particular, we have
\[
A_0(x)
\;=\;
\sum_{k=1}^\infty
  \hat{A}_{0,k}\,\phi_k(x)
\quad\text{with}\quad
\sum_{k=1}^\infty
  \lambda_k^s\,
  \lvert \hat{A}_{0,k}\rvert^2
\,<\,\infty,
\quad
\phi_k\big|_{\partial\Omega}=0.
\]
Then we set 
\(\hat{A}_{0,k}=\langle A_0,\phi_k\rangle_{L^2(\Omega)}\), giving a well-defined expansion.

\paragraph{Goal: Well-Posedness.}
Our objective is to show that, under these hypotheses
\((\alpha\in(0,1),\, s\in(0,1),\, \Omega \subset \mathbb{R}^n \text{ bounded},\, F \text{ globally Lipschitz},\, A_0 \in H_0^s(\Omega))\),
there exists a \textbf{unique solution}
\[
A(t,\cdot) \;\in\; L^\infty\bigl(0,T;\,H_0^s(\Omega)\bigr)
\]
solving \eqref{eq:fixedOrderPDE} in a suitable \emph{weak sense} (see
Definition~\ref{def:weak_soln_extended}).  Furthermore, we will derive energy estimates
to show that \(A\) remains finite for \(t\in[0,T]\).  This approach leverages:

\begin{itemize}
\item A \textbf{Galerkin approximation} using the eigenfunctions
  \(\{\phi_k\}\) of \((-\Delta)\),
\item A \textbf{fractional product rule} and \textbf{fractional Grönwall inequality}
  to obtain uniform (in \(N\)) bounds for the approximate solutions,
\item A \textbf{fractional Volterra compactness argument} (Aubin--Lions--Simon–type lemma)
  to pass \(N\to\infty\) and obtain a limit \(A\),
\item A \textbf{uniqueness} argument based on Lipschitz continuity of \(F\).
\end{itemize}

\begin{remark}[Extensions Beyond Lipschitz]
If \(F\) is only \emph{one-sided} Lipschitz or merely monotone, similar well-posedness
results hold, though one sometimes needs modified uniqueness arguments (e.g.\ Kato-type
inequalities).  If \(F\) grows faster than linearly, blowups can occur in finite time.
See, e.g., \cite[Ch.~7]{Diethelm2010} for classical ODE analogs, adapted to PDEs in
\cite{Zacher2005,Ye2007}.
\end{remark}

This PDE \eqref{eq:fixedOrderPDE} is the prototypical \emph{time-fractional diffusion
equation} with a fixed fractional order \(\alpha\) and spatial exponent \(s\).  
Subsequent sections generalize to \(\alpha(t)\) variable over time, amplitude-triggered
switching of exponents, and eventually an \(\Omega\times\mathcal{A}\) extended manifold
setting.  But the analysis here (Sections~\ref{subsec:galerkin_singledomain}--\ref{subsec:limit_singledomain})
provides the core blueprint for all of those scenarios.

%---------------------------------------------------------------------------
\subsection{Galerkin Approximation and Uniform Energy Bounds}
\label{subsec:galerkin_singledomain}
%---------------------------------------------------------------------------

In order to solve the time-fractional PDE \eqref{eq:fixedOrderPDE} in the single domain
\(\Omega\), we employ a \emph{Galerkin} approach, leveraging the spectral fractional
Laplacian basis \(\{\phi_k\}\) introduced earlier.  This method follows the classical
outline in \cite[Chapter~7]{Diethelm2010} and \cite[Section~3.3]{Kilbas2006}, adapted
to PDEs using the eigenfunction expansion and the Caputo derivative.

\subsubsection{Eigenfunction Truncation.}
Let \(\{\phi_k\}_{k=1}^\infty\) be the Dirichlet eigenfunctions of \(-\Delta\), satisfying
\[
(-\Delta)\,\phi_k \;=\; \lambda_k\,\phi_k,
\quad
\phi_k\big|_{\partial\Omega}=0,
\quad
k=1,2,\dots
\]
Then \(\{(-\Delta)^s \phi_k\} = \{\lambda_k^s \phi_k\}\).  We pick the first \(N\)
eigenfunctions \(\phi_1,\dots,\phi_N\) and form an \emph{approximate solution}
\[
A_N(t,x)
\;=\;
\sum_{k=1}^N a_k^N(t)\,\phi_k(x).
\]
This approximation ensures \(A_N(t,\cdot)\) is always in the span of
\(\{\phi_1,\dots,\phi_N\}\subset H_0^s(\Omega)\).

\subsubsection{Projection onto Each Mode.}
We now \emph{project} the PDE \eqref{eq:fixedOrderPDE} onto each eigenfunction \(\phi_j\).
Concretely, multiply both sides of
\[
\partial_t^\alpha A(t,x) \;+\; (-\Delta)^s A(t,x) \;=\; F\bigl(A(t,x)\bigr)
\]
by \(\phi_j(x)\) and integrate over \(x\in\Omega\). Recall that
\(\int_\Omega \phi_j(x)\,\phi_k(x)\,dx = \delta_{jk}\).  Then we express \(A(t,x)\)
by \(A_N(t,x)\) in \emph{finite-dimensional} form.  Since
\[
(-\Delta)^s A_N(t,x)
\;=\;
\sum_{k=1}^N \lambda_k^s\,a_k^N(t)\,\phi_k(x),
\]
we get, for \(j=1,\dots,N\),
\[
\int_\Omega
  \phi_j(x)\,\partial_t^\alpha A_N(t,x)
\,dx
\;+\;
\int_\Omega
  \phi_j(x)\,
  \Bigl(\sum_{k=1}^N \lambda_k^s\,a_k^N(t)\,\phi_k(x)\Bigr)\,dx
\;=\;
\int_\Omega
  \phi_j(x)\,F\bigl(A_N(t,x)\bigr)\,dx.
\]

Because
\(\displaystyle \int_\Omega \phi_j(x)\,\phi_k(x)\,dx = \delta_{jk}\),
the second tSRFT becomes
\(\lambda_j^s\,a_j^N(t)\). For the first term, we note that
\[
\int_\Omega
  \phi_j(x)\,\partial_t^\alpha A_N(t,x)
\,dx
\;=\;
\partial_t^\alpha\Bigl[\,a_j^N(t)\Bigr]
\quad
(\text{by linearity of }\partial_t^\alpha \text{ and orthonormality}),
\]
assuming we can exchange the Caputo derivative and spatial integration
(see \cite[Sec.~2.3]{Diethelm2010}). Hence we obtain a system of
\textbf{Caputo ODEs}:
\[
\partial_t^\alpha a_j^N(t)
\;+\;
\lambda_j^s\,a_j^N(t)
\;=\;
\int_\Omega
  \phi_j(x)\,F\bigl(A_N(t,x)\bigr)
\,dx,
\quad
j=1,\dots,N.
\]
The \textbf{initial data} follows from expanding \(A_0\in H_0^s(\Omega)\):
\[
a_j^N(0)
= \int_\Omega A_0(x)\,\phi_j(x)\,dx
= \bigl\langle A_0,\phi_j\bigl\rangle_{L^2(\Omega)}.
\]

\subsubsection{Existence and Uniqueness of the Finite ODE System.}
Since \(F\) is globally Lipschitz, the right-hand side
\(\displaystyle
R_j^N(t) := \int_\Omega \phi_j(x)\,F\bigl(A_N(t,x)\bigr)\,dx
\)
is at most linearly growing in \(\lvert A_N\rvert\). Existence and uniqueness for each
\emph{Caputo ODE} in the system,
\[
\partial_t^\alpha a_j^N(t)
+ \lambda_j^s \,a_j^N(t)
= R_j^N(t),
\quad
j=1,\dots,N,
\]
follows directly from the theory of Caputo ODEs with Lipschitz forcing
\cite[Section~2.3]{Diethelm2010}. Thus, for each fixed \(N\), there is a unique solution
\(\{a_j^N(t)\}_{j=1}^N\in C([0,T];\mathbb{R}^N)\).

\subsubsection{Energy Estimate.}
\label{sec:energy_estimates}
To show \(\{A_N\}\) is uniformly bounded in \(H_0^s(\Omega)\), we define the
\textbf{energy} of the approximate solution:
\[
E_N(t)
\;:=\;
\|A_N(t)\|_{H_0^s(\Omega)}^2
\;=\;
\sum_{j=1}^N
  \lambda_j^s
  \,\bigl(a_j^N(t)\bigr)^2.
\]
We want a \emph{differential inequality} in terms of the Caputo derivative
\(\partial_t^\alpha E_N(t)\). One standard way is:
\[
\begin{aligned}
\partial_t^\alpha\bigl[a_j^N(t)\bigr]
&=
-\lambda_j^s\,a_j^N(t)
+ R_j^N(t),
\\[6pt]
\text{multiply by } \lambda_j^s\,a_j^N(t)
&\implies
\lambda_j^s\,a_j^N(t)\,\partial_t^\alpha \bigl[a_j^N(t)\bigr]
=
- \lambda_j^{2s}\,\bigl(a_j^N(t)\bigr)^2
+ \lambda_j^s\,a_j^N(t)\,R_j^N(t).
\end{aligned}
\]
Summing over \(j=1,\dots,N\):
\[
\sum_{j=1}^N
  \lambda_j^s\,a_j^N \,\partial_t^\alpha a_j^N
=
-\sum_{j=1}^N
  \lambda_j^{2s}\,\bigl(a_j^N\bigr)^2
+ \sum_{j=1}^N
  \lambda_j^s\,a_j^N\,R_j^N.
\]
Using the \textbf{fractional product rule} 
(Lemma~\ref{lem:fr_product_rule_expanded} or \cite[Thm.~2.2]{Diethelm2010}) or
an argument that lumps everything into a single “multiply-by-\(A_N\)” step, one obtains
\[
\partial_t^\alpha
  \Bigl[
    \tfrac12
    \sum_{j=1}^N
      \lambda_j^s
      \,\bigl(a_j^N(t)\bigr)^2
  \Bigr]
=
\partial_t^\alpha
  \Bigl[
    \tfrac12\,E_N(t)
  \Bigr]
=
\sum_{j=1}^N
  \lambda_j^s\,a_j^N\,\partial_t^\alpha a_j^N
\quad
\text{(possibly plus a boundary term, which is zero under typical assumptions).}
\]
Hence,
\[
\partial_t^\alpha E_N(t)
=
-2
\sum_{j=1}^N
   \lambda_j^{2s}\,\bigl(a_j^N\bigr)^2
+ 2
\sum_{j=1}^N
  \lambda_j^s\,a_j^N\,R_j^N.
\]
The negative sum \(\sum_{j=1}^N \lambda_j^{2s}\ldots\) can be bounded below by
\(-C_1\,E_N\), and \(\sum_{j=1}^N \lambda_j^s\,a_j^N\,R_j^N\) can be bounded by
\(C_2 + C_1E_N\) due to the global Lipschitz nature of \(F\). Altogether, one arrives at
an inequality of the form
\[
\partial_t^\alpha E_N(t)
\;\le\;
C_1\,E_N(t) + C_2,
\]
where \(C_1,C_2\) depend on the Lipschitz constant \(L_F\), the eigenvalues \(\lambda_j\),
and possibly \(\|A_0\|_{H_0^s(\Omega)}\).

\subsubsection{Applying the Fractional Grönwall Inequality.}
We invoke a **fractional Grönwall** argument. For instance, in the constant-\(\alpha\)
case (\(\alpha\in(0,1)\) fixed), if we have
\[
\partial_t^\alpha
  E_N(t)
\;\le\;
a\,E_N(t)
\;+\;
b,
\quad
E_N(0)=E_N(0)\ge0,
\]
then by \cite[Section~3.1]{Diethelm2010} or \cite{Ye2007}, we derive
\[
E_N(t)
\;\le\;
\Bigl(E_N(0) + \frac{b}{a}\Bigr)
\,E_\alpha\bigl(a\,t^\alpha\bigr)
\;-\;
\frac{b}{a},
\]
where \(E_\alpha\) is the Mittag-Leffler function.  Hence \(E_N(t)\) remains uniformly
bounded for \(t\in[0,T]\).  Consequently,
\[
A_N(t,\cdot)
\;\in\;
L^\infty\bigl(0,T;\,H_0^s(\Omega)\bigr)
\quad
\text{with bounds independent of }N.
\]

\begin{remark}[Variable \(\alpha(t)\)]
If \(\alpha\) is time-varying but bounded away from 0 and below 1,
\(\alpha_{\min}\le \alpha(t)\le\alpha_{\max}<1\), an extended “Volterra stability” or
piecewise-constant approximation argument (see \cite{Sun2019,Ye2007}) shows a similar
fractional Grönwall lemma applies. The result is still that \(E_N(t)\) is uniformly
bounded.  
\end{remark}

\subsubsection{Conclusion: Uniform Boundedness.}
Thus, we have constructed a sequence \(\{A_N\}_{N=1}^\infty\) of approximate solutions
satisfying:
\[
\bigl\|A_N(t)\bigr\|_{H_0^s(\Omega)}^2
\;=\;
E_N(t)
\;\le\;
C,
\quad
\text{for all } t\in [0,T],
\]
with \(C\) independent of \(N\).  This uniform (in \(N\)) bound is the essential step
for applying compactness arguments (Aubin--Lions--Simon type) in the next subsection,
where we will let \(N\to\infty\) to extract a weak limit that solves the PDE
\eqref{eq:fixedOrderPDE}.
%---------------------------------------------------------------------------
\subsection{Passing to the Limit in \texorpdfstring{(\(N\))}: Weak Compactness}
\label{subsec:limit_singledomain}
%---------------------------------------------------------------------------

We have shown in Section~\ref{subsec:galerkin_singledomain} that the sequence
\(\{A_N\}\) of Galerkin approximations satisfies a uniform bound:
\[
\sup_{N}\,
\bigl\|A_N(t)\bigr\|_{H_0^s(\Omega)}
\;\le\;
C,
\quad
\text{for all } t\in[0,T].
\]
Thus, 
\[
A_N \;\in\; L^\infty\!\bigl(0,T;\,H_0^s(\Omega)\bigr),
\]
with a bound independent of \(N\). We now aim to extract a subsequence
\(\{A_{N_k}\}\) that converges (weakly) to some \(A\) in a suitable function space,
and then identify the limit as a solution of \eqref{eq:fixedOrderPDE}.

\subsubsection{Weak Convergence in \texorpdfstring{$H_0^s(\Omega)$}{H0s(Omega)}}
\label{sec:partA_weak_form}
By Banach–Alaoglu (or equivalently the uniform boundedness in \(L^\infty\bigl(0,T;\,H_0^s(\Omega)\bigr)\)),
there exists a subsequence \(\{A_{N_k}\}\) and a function
\[
A \;\in\; L^\infty\bigl(0,T;\,H_0^s(\Omega)\bigr)
\]
such that, for almost every \(t\in[0,T]\),
\[
A_{N_k}(t)\;\rightharpoonup\; A(t)
\quad
\text{weakly in }H_0^s(\Omega),
\]
as \(k\to\infty\). That is, for all \(\phi\in H_0^s(\Omega)\),
\[
\int_\Omega
  A_{N_k}(t,x)\,\phi(x)\,dx
\;\to\;
\int_\Omega
  A(t,x)\,\phi(x)\,dx.
\]
The uniform bound ensures that each \(A_N\) remains in a weakly relatively compact set.

\subsubsection{Fractional Volterra / Caputo Derivative Compactness.}
A core difficulty in fractional PDE analysis is ensuring that the \textbf{Caputo derivative}
\(\partial_t^\alpha A_N\) also converges appropriately.  Recall the Caputo operator is a
\emph{Volterra integral operator} acting on \(A_N'(t)\), formally:
\[
\partial_t^\alpha A_N(t)
=
\frac{1}{\Gamma(1-\alpha)}
\int_0^t
  \frac{\partial_\tau A_N(\tau)}{(t-\tau)^\alpha}\,d\tau.
\]
To pass to the limit in \(\partial_t^\alpha A_N\), we need a version of
the \textbf{Aubin–Lions–Simon lemma} adapted to fractional evolution.  Several results in
the literature \cite[Ch.~4]{Diethelm2010}, \cite[Section~4.1]{Kilbas2006}, and
\cite{Simon1987} show that if \(\{A_N\}\) is suitably bounded in
\(L^\infty(0,T;H_0^s(\Omega))\) and their “fractional time derivative” is controlled
in an appropriate dual or negative norm, then \(\{A_N\}\) is relatively compact in a
weaker topology.  The main point is that \(\partial_t^\alpha A_N\) can be seen as a
Volterra convolution in time, so bounding it yields equicontinuity (or equicontrolledness)
in a suitable sense.  (See also \cite{Sun2019} for the piecewise extension if \(\alpha\)
is variable or amplitude‐triggered.)

Concretely, one shows \(\|\partial_t^\alpha A_N\|_{X}\le C\) for some Banach space \(X\)
that is the dual of a certain subspace of \(H_0^s(\Omega)\).  Then an argument akin to
Aubin–Lions or a Volterra stability principle ensures that a subsequence of \(A_N\)
is \emph{strongly} convergent in lower‐dimensional topologies or \emph{weakly} convergent
in the full sense.  For the present PDE, we only need a \emph{weak} or \emph{weak*}
convergence to identify the limit PDE.

\subsubsection{Identifying the Limit in the PDE.}
To see that the limit \(A\) satisfies
\[
\partial_t^\alpha A + (-\Delta)^s A = F(A),
\quad
A(0)=A_0,
\]
we note:

\begin{itemize}
\item \emph{Weak Convergence of \((- \Delta)^s A_{N_k}\).}
  Since \(A_{N_k}(t)\rightharpoonup A(t)\) in \(H_0^s(\Omega)\), one has
  \[
  (-\Delta)^s A_{N_k}(t)
  \;=\;
  \sum_{j=1}^N \lambda_j^s\,a_j^{N_k}(t)\,\phi_j
  \;\rightharpoonup\;
  \sum_{j=1}^\infty \lambda_j^s \,\hat{A}_j(t)\,\phi_j
  \;=\;
  (-\Delta)^s A(t),
  \]
  in the weak sense of \(H_0^s(\Omega)\).  (One can also see this as
  \(\|(-\Delta)^s(A_{N_k}-A)\|\to 0\) if we have a strong or norm convergence, but
  weak or distributional is enough to identify the limit in the PDE.)
\item \emph{Weak Convergence of \(\partial_t^\alpha A_{N_k}\).}
  By the fractional compactness (Volterra argument), we obtain
  \[
  \partial_t^\alpha A_{N_k}(t)
  \;\rightharpoonup\;
  \partial_t^\alpha A(t)
  \quad
  \text{in a dual or negative norm sense.}
  \]
  So when projecting the PDE onto test functions (or in the distribution sense in time),
  \(\partial_t^\alpha A_{N_k}\to \partial_t^\alpha A\).
\item \emph{Convergence of \(F(A_{N_k})\).}
  Since \(F\) is globally Lipschitz, \(\{A_{N_k}\}\to A\) weakly in \(H_0^s(\Omega)\)
  also implies \(F(A_{N_k})\to F(A)\) in at least a weak or strong sense in \(L^2(\Omega)\).
  Typically, the Lipschitz continuity and boundedness arguments suffice to identify
  \(F(A_{N_k})\rightharpoonup F(A)\).  
\end{itemize}

Hence, each tSRFT in the finite-dimensional ODE system converges to the corresponding term
in the PDE \(\partial_t^\alpha A + (-\Delta)^s A = F(A)\).  Finally, we verify the initial
condition by checking that \(\|A_{N_k}(0)-A_0\|_{H_0^s(\Omega)}\to0\) as \(N_k\to\infty\),
which holds by construction of the Galerkin initial data \(a_j^N(0)=\langle A_0,\phi_j\rangle\).

\subsubsection{Conclusion: Existence and Uniqueness.}
Summarizing, we have:

\begin{theorem}[Existence and Uniqueness: Fixed‐Order Fractional PDE]
\label{thm:FixedOrderExist_expanded}
Let \(0<\alpha<1\), \(0<s<1\), and let \(F:\mathbb{R}\to \mathbb{R}\) be \textbf{globally Lipschitz}.
Assume \(A_0\in H_0^s(\Omega)\). Then there exists a function
\[
A \;\in\; L^\infty\bigl(0,T;\,H_0^s(\Omega)\bigr)
\]
such that \(A\) is a \emph{weak solution} of the time‐fractional PDE
\[
\partial_t^\alpha A(t,x) 
\;+\; 
(-\Delta)^s A(t,x)
\;=\;
F\bigl(A(t,x)\bigr),
\]
subject to the Dirichlet boundary condition \(A(t,x)\big|_{x\in \partial\Omega}=0\) 
and the initial condition \(A(0,x) = A_0(x)\). 

Uniqueness follows from a \textbf{fractional Grönwall‐type estimate} applied
to the difference of any two solutions (see \cite[Ch.~7]{Diethelm2010}
for the standard Caputo ODE argument, adapted to PDEs in \cite{Zacher2005,Ye2007}).
\end{theorem}

\begin{remark}[One‐Sided Lipschitz or Monotone \(F\)]
If \(F\) is not globally Lipschitz but only monotone or one‐sided Lipschitz, there are
still well‐posedness results under additional constraints (see
\cite[Ch.~7]{Diethelm2010}).  We assume full Lipschitz here for simplicity, ensuring
straightforward uniqueness via a fractional Grönwall approach.
\end{remark}

\smallskip

Thus, we have established \emph{existence and uniqueness} for the single-domain fractional
PDE
\(\partial_t^\alpha A + (-\Delta)^s A = F(A)\) with fixed order \(\alpha\in(0,1)\).
In subsequent sections, we explore variations (variable‐order \(\alpha(t)\), amplitude
threshold switching, and eventually an extended domain \(\Omega\times\mathcal{A}\) in
Part~D).

%=========================================================
\section{Part B: Variable‐Order in Time}
\label{sec:var_order}
%=========================================================

Having established the well-posedness of the fixed-order fractional PDE in Part~A, 
we now allow the time-fractional order to vary with time. In this part, we consider 
$\alpha=\alpha(t)$, which lies in a subinterval of $(0,1)$, and develop a piecewise 
approximation approach to handle the variable-order Caputo derivative. This enables us 
to extend the results of Part~A to cases where memory effects evolve over time.

In \textbf{Part~A}, we analyzed the existence and uniqueness of solutions to a
\emph{time-fractional PDE} in which the order $\alpha\in(0,1)$ was a \emph{constant}.
While that setting already encompasses subdiffusive or long-memory behavior, many
practical and theoretical studies require the \emph{fractional order} to
\emph{evolve over time}---reflecting, for instance, changes in the medium's properties,
the system's memory capacity (see Remark~\ref{rem:memory_reinit}), or other physical parameters that vary during the process. 

Hence, in \textbf{Part~B}, we turn to the \textbf{variable-order} scenario, denoting
\[
\alpha=\alpha(t)\in \bigl[\alpha_{\min},\alpha_{\max}\bigr]\subset(0,1).
\]
Here, $\alpha(t)$ is no longer a fixed constant but a \emph{continuous} function of
$t\in[0,T]$. This modification induces important questions about whether the PDE
remains well-posed---can we still construct solutions through Galerkin approximations?
Does the energy estimate (fractional Grönwall) carry over to variable exponents? And how
do we handle potential discontinuities or piecewise definitions of $\alpha(t)$?

\paragraph{Major Themes and Goals.}
\begin{itemize}
    \item \emph{Volterra Stability Principle.} When $\alpha(t)$ varies,
          the Caputo derivative $\partial_t^{\alpha(t)}$ becomes a \emph{variable-kernel}
          convolution operator in time. We will exploit \emph{Volterra stability results}
          that guarantee small changes in the kernel still lead to small changes in
          solutions.
    \item \emph{Piecewise-Constant Approximation of \(\alpha(t)\).} A common approach is
          to approximate $\alpha(t)$ by step functions (or simple polynomials) on a time
          partition. Then on each sub-interval, $\alpha$ is ``frozen'' at a fixed value,
          reducing the problem to the \emph{fixed-order PDE} from Part~A. By matching
          solutions across interval boundaries, we obtain a piecewise solution that
          converges to the \emph{true} variable-order solution.
    \item \emph{Fractional Grönwall with Variable \(\alpha(t)\).} We will invoke a
          \emph{variable-order} version of the fractional Grönwall lemma to ensure
          energy bounds remain uniform in time. This is essential for passing to the
          limit in Galerkin schemes.
    \item \emph{Uniqueness under Lipschitz or Monotone \(\mathbf{F}\).} Finally, to
          demonstrate uniqueness, we adapt the fractional Grönwall approach to show that
          if two solutions existed, their difference must remain zero, provided $F$ is
          Lipschitz (or at least one-sided Lipschitz).
\end{itemize}

\paragraph{Outline.}
In the upcoming \textbf{Preliminaries and Notation} subsection, we reiterate the definitions
for the variable-order Caputo derivative and the assumptions on $\alpha(t)$, referencing
key lemmas from \cite{Sun2019,Ye2007}. We then proceed to the main \emph{existence proof}:
we approximate $\alpha(t)$ by piecewise-constant functions $\alpha_M(t)$, solve a
fixed-order PDE on each sub-interval, and pass to the limit via compactness. Uniqueness
follows from a variable-order difference-of-solutions argument. Thus, we extend the
\emph{single-domain} well-posedness from Part~A to the more general, continuously evolving
exponent $\alpha(t)$.
%---------------------------------------------------------------------------
\subsection{Preliminaries and Notation (Part B: Variable-Order PDE)}
\label{subsec:prelim_partB}
%---------------------------------------------------------------------------

In this subsection, we collect the main definitions and assumptions for 
the \textbf{variable‐order time‐fractional PDE} scenario. We build upon the 
setup from Part~A but allow the fractional exponent \(\alpha\) to depend on time 
\(t\). We also specify how to apply fractional Gr\"onwall lemmas when 
\(\alpha(t)\) is not constant.

\subsubsection{Domain, Fractional Laplacian, and Function Spaces (Same as Part~A)}

As in Part~A, we let \(\Omega\subset \mathbb{R}^n\) be a bounded domain with 
sufficiently regular boundary \(\partial\Omega\) (Lipschitz or \(C^{1,\alpha}\)). 
We use the \textbf{spectral fractional Laplacian} \((- \Delta)^s\) of order 
\(s\in(0,1)\), imposing \emph{Dirichlet boundary conditions}, i.e.\ 
\[
A(t,x)\big|_{\partial\Omega} = 0 
\quad\text{in a fractional‐trace sense.}
\]
For $u\in L^2(\Omega)$ with expansion $u(x)=\sum_{k=1}^\infty \hat{u}_k\,\phi_k(x)$ 
in the Dirichlet eigenfunctions of $-\Delta$, we define
\[
(-\Delta)^s u
=
\sum_{k=1}^\infty \lambda_k^s\,\hat{u}_k\,\phi_k(x),
\]
and let $H_0^s(\Omega)$ be the associated energy space as in Part~A.

\subsubsection{Variable‐Order Caputo Derivative \texorpdfstring{\(\alpha(t)\in(0,1)\)}{}}

Here, the time‐fractional exponent $\alpha$ is \emph{not} fixed but a function 
\[
\alpha(t)\in \bigl[\alpha_{\min},\,\alpha_{\max}\bigr]\subset (0,1),
\]
assumed continuous on $[0,T]$. Then the \emph{variable‐order Caputo derivative} 
can be informally written as
\[
\partial_t^{\alpha(t)} A(t)
=
\frac{1}{\Gamma\bigl(1-\alpha(t)\bigr)}
\int_0^t
  \frac{\partial_\tau A(\tau)}{\bigl(t-\tau\bigr)^{\alpha(t)}}
\,d\tau,
\]

though rigorous definitions often rely on piecewise approximations 
(\cite{Sun2019}, \cite[Ch.~6]{Magin2006}). The essential point is that 
\(\alpha(\cdot)\) can change over time, but remains in \((0,1)\). We still require 
$A$ to be sufficiently smooth (or piecewise $C^1$) so that the integral converges.

\begin{remark}[Piecewise‐Constant Approximation of \(\alpha(t)\)]
\label{rem:alpha_piecewise}
A standard technique is to approximate \(\alpha(t)\) by a sequence 
$\{\alpha_M(t)\}$ that is \emph{piecewise constant} on a partition 
$0=t_0<t_1<\dots<t_{p}=T$. On each subinterval $[t_{m-1},t_m]$, 
$\alpha_M(t)=\alpha_m$ (constant). One solves a \textbf{fixed‐order} PDE 
separately on each subinterval and matches solutions at $t_m$. 
Letting $M\to \infty$ (finer partition) yields a limit that solves the 
variable‐order PDE.  We will rely on this approach in Part~B to prove 
existence and uniqueness.
\end{remark}

\subsubsection{Variable‐Order Fractional Gr\"onwall}

To obtain uniform energy estimates under \(\alpha(t)\), one uses a 
\textbf{variable‐order fractional Gr\"onwall} lemma.  If $y(t)\ge0$ satisfies
\[
\partial_t^{\alpha(t)}\,y(t) \;\le\; a\,y(t) + b,
\quad
\alpha(t)\in[\alpha_{\min},\alpha_{\max}]\subset(0,1),
\]
then $y(t)$ remains bounded on $[0,T]$.  Several versions of this result 
appear in \cite{Sun2019}, \cite{Ye2007}, based on a \emph{Volterra stability} 
argument.  The core idea is similar:  the bound on $y(t)$ typically takes 
a generalized Mittag‐Leffler form, ensuring $y$ does not blow up in finite time.

\subsubsection{Notation Summary for Part~B}

\begin{itemize}
\item $\Omega\subset \mathbb{R}^n$: bounded domain, $s\in(0,1)$ for the fractional 
      Laplacian $(-\Delta)^s$ with Dirichlet boundary. 
\item $\alpha(t)\in C\bigl([0,T];[\alpha_{\min},\alpha_{\max}]\bigr)$: continuous time‐fractional 
      exponent, $0<\alpha_{\min}\le \alpha(t)\le \alpha_{\max}<1$. 
\item $\partial_t^{\alpha(t)}$: variable‐order Caputo derivative (approx.\ piecewise 
      constant or polynomial in actual computation).
\item $F:\mathbb{R}\to\mathbb{R}$: again assumed **globally Lipschitz**, for the forcing 
      tSRFT in the PDE. 
\item $A_0\in H_0^s(\Omega)$: initial data. 
\end{itemize}

With these definitions in place, Part~B will present the well‐posedness argument for 
\[
\partial_t^{\alpha(t)} A(t,x)
+ 
(-\Delta)^s\,A(t,x)
=
F\bigl(A(t,x)\bigr),
\quad
A(0,x)=A_0(x),
\quad
A|_{\partial\Omega}=0,
\]
mirroring the fixed‐order approach (Part~A) but using the piecewise‐constant approximation 
of $\alpha(t)$ and a \emph{variable‐order} fractional Gr\"onwall estimate to obtain a 
unique solution in $L^\infty(0,T;H_0^s(\Omega))$.

%---------------------------------------------------------
\subsection{Stepwise Approximation and Volterra Stability}
\label{subsec:stepwise_varalpha}
%---------------------------------------------------------

The core idea is to \emph{approximate} the continuous function \(\alpha(t)\) by
\(\{\alpha_M(t)\}\), a sequence of \emph{piecewise‐constant} (or piecewise polynomial)
functions that converge \emph{uniformly} to \(\alpha(t)\). Then on each subinterval
\([t_{m-1}, t_m]\), \(\alpha_M\) is constant, say \(\alpha_m\). We thereby solve a PDE
with \emph{fixed} exponent \(\alpha_m\) on each subinterval and match the solution at
the boundaries of subintervals. Finally, letting \(M\to\infty\) (so
\(\max_m|t_m-t_{m-1}|\to 0\)) recovers the solution for the original \(\alpha(t)\).

\subsubsection{Piecewise‐Constant Approximation.}
Partition the interval \([0,T]\) as \(0=t_0<t_1<\dots<t_{p-1}<t_p=T\), with
\(\Delta t_m := t_m - t_{m-1}\). Define
\[
\alpha_M(t)
=
\sum_{m=1}^p
  \mathbf{1}_{[t_{m-1}, t_m)}(t)\,\alpha_m,
\quad
\text{where}
\quad
\alpha_m
= \alpha\bigl(\theta_m\bigr),
\quad
\text{for some }\theta_m\in [t_{m-1},t_m].
\]
Hence, \(\alpha_M\) is constant \(\alpha_m\) on each subinterval
\([t_{m-1},t_m)\). By the uniform continuity of \(\alpha(\cdot)\),
\(\|\alpha_M(t) - \alpha(t)\|_{\infty}\to0\) as \(M\to\infty\).

\subsubsection{Solving a PDE with Fixed \(\alpha_m\).}
On each subinterval \([t_{m-1},t_m]\), we solve
\[
\partial_t^{\alpha_m} A_M(t,x)
\;+\;
(-\Delta)^s A_M(t,x)
=
F\bigl(A_M(t,x)\bigr),
\quad
t\in [t_{m-1}, t_m],
\]
with an initial condition at \(t_{m-1}\).  As explained in Part~A, the \emph{fixed-order}
theory applies, ensuring existence and uniqueness of a solution on that subinterval,
provided we carefully set \(A_M(t_{m-1},\cdot)\) as the matching condition from the
previous interval’s final value.  We do this successively for \(m=1,\dots,p\). The result
is a piecewise-defined function \(A_M(t,\cdot)\) on \([0,T]\) that is continuous in time
and satisfies the PDE with exponent \(\alpha_m\) on each subinterval.

\subsubsection{Uniform (in \(M\)) Energy Estimates.}
Exactly as in the fixed-order case, each sub-problem has a \emph{fractional Grönwall}
inequality for the energy
\(\displaystyle
E_M(t) = \|A_M(t)\|_{H_0^s(\Omega)}^2
\).
But now \(\partial_t^{\alpha_m} E_M\le a\,E_M +b\). Over each subinterval
\([t_{m-1}, t_m]\), the same argument ensures a uniform bound on \(E_M(t)\). Concatenating
these subintervals, and using the fact that \(\alpha_m\) is bounded away from 1 and above
0 (since \(\alpha_m\in[\alpha_{\min},\alpha_{\max}]\)), yields a global in time energy
bound on \(\|A_M(t)\|_{H_0^s(\Omega)}\).  

To make this more precise, we invoke a \textbf{variable-order fractional Grönwall}
inequality:

\begin{lemma}[Fractional Grönwall, Variable \(\alpha(t)\)]
\label{lem:FG_VarAlpha_expanded}
Suppose \(y(t)\ge0\) satisfies
\[
\partial_t^{\alpha(t)} y(t)
\;\le\;
a\,y(t) + b,
\quad
t\in(0,T],
\quad
y(0)=y_0\ge0,
\]
where \(\alpha(t)\in [\alpha_{\min},\alpha_{\max}]\subset(0,1)\) is continuous and
\(a,b\ge0\). Then there is a generalized Mittag–Leffler‐type upper bound on \(y(t)\),
depending on \(\alpha_{\min}, \alpha_{\max}, a,\) and \(b\).  In particular, \(y(t)\)
remains uniformly bounded on \([0,T]\). See \cite[Section~3]{Ye2007} or \cite{Sun2019}
for a precise statement and proof using \emph{Volterra stability} arguments.
\end{lemma}

Thus, each approximate solution \(A_M\) remains uniformly bounded in
\(\displaystyle L^\infty\bigl(0,T;H_0^s(\Omega)\bigr)\).

\subsubsection{Passing to the Limit: Volterra Stability and Compactness.}
Having bounded \(\|A_M(t)\|_{H_0^s(\Omega)}\), we next apply the same fractional
Aubin–Lions–Simon type reasoning as in \S\ref{subsec:limit_singledomain}: we ensure that
\(\{\partial_t^{\alpha_M(t)}A_M\}\) is also bounded in an appropriate dual space, or that
the difference \((A_M(t)-A_M(\tau))\) is equicontinuous in time for \(\tau,t\) in subintervals.
Hence, we can extract a subsequence \(A_{M_k}\) converging (weakly or weak*) to some
\(A\in L^\infty(0,T;H_0^s(\Omega))\). By verifying that \(\alpha_M(t)\to\alpha(t)\) 
uniformly and matching PDE terms in the limit, we identify
\[
\partial_t^{\alpha(t)} A + (-\Delta)^s A = F(A),
\quad
A(0)=A_0,
\quad
A|_{\partial\Omega}=0.
\]

The details rely on a \emph{Volterra stability} principle: small changes in the kernel
\((t-\tau)^{-\alpha_M(t)}\) vs. \((t-\tau)^{-\alpha(t)}\) lead to small changes in the
convolution integral, see \cite{Sun2019} for a thorough exposition.

\subsubsection{Conclusion: Existence and Uniqueness.}
Uniqueness follows the same fractional Grönwall approach as before:
if \(A^{(1)},A^{(2)}\) are two solutions with the same data, their difference
\(D=A^{(1)}-A^{(2)}\) satisfies
\[
\partial_t^{\alpha(t)} D + (-\Delta)^s D = F(A^{(1)})-F(A^{(2)}),
\]
and by Lipschitz continuity of \(F\), we get an ODE of the form
\(\partial_t^{\alpha(t)} \|D\|\le a\,\|D\|\), so \(\|D\|\equiv0\). We thus obtain:

\begin{theorem}[Existence and Uniqueness, Variable \(\alpha(t)\)]
\label{thm:VarOrderExist_expanded}
Let \(\alpha(t)\in C\bigl([0,T];[\alpha_{\min},\alpha_{\max}]\bigr)\subset(0,1)\) and let
\(F:\mathbb{R}\to\mathbb{R}\) be globally Lipschitz. Suppose \(A_0\in H_0^s(\Omega)\). Then the
variable‐order PDE \eqref{eq:varOrderPDE} admits a unique solution
\[
A \;\in\; L^\infty\bigl(0,T;H_0^s(\Omega)\bigr).
\]
Specifically,
\begin{equation}
\label{eq:varOrderPDE}
\partial_t^{\alpha(t)} A(t,x)
+ 
(-\Delta)^s\,A(t,x)
=
F\bigl(A(t,x)\bigr),
\quad
A(0,x)=A_0(x),
\quad
A\big|_{\partial\Omega}=0.
\end{equation}


in the weak sense.  The proof uses the piecewise‐constant approximation of \(\alpha(t)\),
fractional Grönwall estimates for uniform bounds, and a Volterra stability/compactness
argument to pass to the limit.

Furthermore, \(A\in L^\infty(0,T;H_0^s(\Omega))\) satisfies the PDE in the weak sense, with 
Dirichlet boundary condition \(A|_{\partial\Omega} = 0\) and initial condition \(A(0)=A_0\). 
Uniqueness is ensured by a similar fractional Grönwall argument on the difference of solutions.

\end{theorem}

\begin{remark}[Amplitude‐Triggered Switching of \(\alpha(t)\)]
Even if \(\alpha\) changes upon amplitude crossing some threshold, the same piecewise
logic applies: on each sub‐interval of time \([t_{m-1},t_m]\), \(\alpha\) is taken as
constant \(\alpha_{\mathrm{low}}\) or \(\alpha_{\mathrm{high}}\). As we refine the
partition, we approximate the amplitude triggers. This ensures existence; uniqueness
again follows from monotonicity or Lipschitz arguments. See \cite{Sun2019} for further
discussion on piecewise variable exponents in fractional PDEs.
\end{remark}

Hence, **variable-order** fractional PDEs on \(\Omega\) remain well-posed under the same
key hypotheses: \(\alpha(t)\in[\alpha_{\min},\alpha_{\max}]\subset(0,1)\), globally
Lipschitz \(F\), and \(A_0\in H_0^s(\Omega)\). The entire Galerkin + fractional Grönwall
+ memory reinitialization approach carries over, with only minor modifications in the
subinterval-by-subinterval analysis.
%=========================================================
%=========================================================
\section{Part C: SRFT Framework and Blowup Regimes}
\label{sec:erm_blowup}
%=========================================================

Having established the single-domain well-posedness for both \emph{fixed-order} and \emph{variable-order} fractional PDEs in Parts~A and~B, we now incorporate \textbf{amplitude-triggered switching} in the \textbf{Self Referential Field Theory (SRFT)} sense. In this \emph{Part~C}, the fractional exponents (\(\alpha\) and/or \(s\)) and possibly the forcing \(F\) can abruptly change whenever the solution amplitude \(\lvert A(t,x)\rvert\) crosses a prescribed threshold \(A_{\mathrm{crit}}\). This allows for the possibility of \emph{finite-time blowups}, \emph{high-amplitude lumps}, or \emph{saturated regimes}, reflecting more complex behaviors than in Parts~A--B.

\paragraph{Motivation and Basic Idea.}
In physical and biological systems, certain processes undergo a qualitative change once the solution (e.g., temperature, concentration, or stress) reaches a critical amplitude. This can induce abrupt memory changes, more (or less) intense diffusion, or new forcing terms. The SRFT viewpoint encapsulates these effects by splitting \(\alpha\) or \(s\) (and possibly \(F\)) into \emph{low} and \emph{high} regimes triggered by an amplitude threshold. Consequently, the PDE may transition from a “low-amplitude fractional PDE” to a “high-amplitude fractional PDE,” or vice versa.

\begin{remark}[Amplitude vs.\ Amplitude Coordinate]
In Parts~A--C, “amplitude” refers to the magnitude $\lvert A(t,x)\rvert$ in the single domain.
By contrast, in Part~D we treat “amplitude” as an \emph{independent coordinate} $a\in\mathcal{A}$
within the extended manifold $\mathcal{M}=\Omega\times\mathcal{A}$. 
Despite this difference, the same threshold logic (crossing a critical level and switching 
exponents or forcing) applies in both settings.
\end{remark}

\paragraph{Key Objectives of Part~C.}
\begin{itemize}
    \item \textbf{Threshold Time \(\boldsymbol{t_b}\) and Piecewise PDE Definition}: We define the first time \(t_b\) at which \(\max_{x\in\Omega}\lvert A(t,x)\rvert\) hits \(A_{\mathrm{crit}}\). For \(t<t_b\), the PDE uses “low” exponents and forcing; for \(t>t_b\), we switch to “high” exponents and forcing. We ensure continuity at \(t_b\) and handle memory reinitialization or continuation in the Caputo derivative.
    \item \textbf{Potential Blowup or Saturation}: 
    If the “high” PDE regime truly drives unbounded growth, amplitude might blow up in finite time, ending the classical solution at a blowup time \(t_{\mathrm{bu}}<T\). Alternatively, if \(F_{\mathrm{high}}\) saturates amplitude, lumps form but remain finite. We unify these scenarios under a piecewise existence theorem.
    \item \textbf{Uniqueness vs. Non-Lipschitz Forces}: 
    If \(F_{\mathrm{high}}\) remains Lipschitz or monotone, uniqueness often persists via fractional Grönwall arguments. But if \(F_{\mathrm{high}}\) is strongly nonlinear (e.g., superlinear and non-monotone), uniqueness may fail and only an existence result holds.
\end{itemize}

\paragraph{Relation to Parts~A and~B.}
Technically, each \emph{sub-interval} $[0,t_b]$ or $[t_b,T]$ can be viewed as a \emph{fixed-order} or \emph{variable-order} PDE from Parts~A--B, with constant or time-dependent \(\alpha\). The new feature is that amplitude crossing a threshold \(\lvert A\rvert=A_{\mathrm{crit}}\) triggers a \textbf{regime switch} in \(\alpha\), \(s\), or $F$. Thus, the same Galerkin + fractional Grönwall + limit passage arguments apply \emph{piecewise} in time . The continuity or jump conditions at $t_b$ are handled by matching $A(t_b^+)=A(t_b^-)$ and deciding on memory reinitialization.

\paragraph{Outline of Part~C.}
\begin{itemize}
    \item \emph{Preliminaries and Notation}: We restate the amplitude‐triggered logic, splitting $\alpha,\ s,\ F$ into “low” vs. “high” definitions. We also discuss memory reinitialization and blowup vs. saturation scenarios.
    \item \emph{Piecewise PDE and Threshold Crossing}: We define the threshold time $t_b$ at which amplitude first hits $A_{\mathrm{crit}}$. Before $t_b$, the PDE is in the low regime; after $t_b$, it is in the high regime. A continuity condition on $A(t_b)$ ensures a well-defined piecewise solution.
    \item \emph{Well-Posedness, Blowup, or Lumps}: If $F_{\mathrm{high}}$ is Lipschitz, a local or global solution persists until a blowup time or final time $T$. If blowup occurs, we terminate classical solutions. If lumps saturate, amplitude remains finite. Theorems detail each scenario.
    \item \emph{Uniqueness Issues}: We highlight that if $F_{\mathrm{high}}$ is not Lipschitz or monotone, uniqueness can fail. We cite classical Caputo ODE results from \cite[Ch.~7]{Diethelm2010} adapted to PDEs.
\end{itemize}

\noindent
Hence, \textbf{Part~C} extends the single-domain fractional PDE theory to \emph{amplitude-triggered multi-regime} settings, encompassing threshold crossing times, memory reinitialization vs.\ global memory, and potential blowups or lumps. With these foundations, we then proceed to the \textbf{Preliminaries and Notation} specific to the SRFT viewpoint and blowup analysis.

%---------------------------------------------------------------------------
\subsection{Preliminaries and Notation (Part C: SRFT Framework and Blowup Regimes)}
\label{subsec:prelim_partC}
%---------------------------------------------------------------------------

In \textbf{Part~C}, we extend the single-domain fractional PDE analysis to incorporate
\emph{amplitude-triggered} changes in the fractional exponents (or in the forcing term
\(F\)), following the \textbf{Self Referential Field Theory (SRFT)} viewpoint.
We focus on how threshold crossings can cause \(\alpha(x,A)\) or \(s(x,A)\) to switch
from “low” to “high” values, potentially enabling \emph{finite-time blowups} or stable
amplitude “lumps.” Below, we outline key definitions and assumptions we shall use.

\subsubsection{Amplitude-Dependent Exponents and Threshold Logic}
\label{subsubsec:amplitude_exponents}

We suppose that each of \(\alpha,\ s,\) and \(F\) may split into \emph{low} vs.\ 
\emph{high} regimes depending on the solution amplitude \(\lvert A(t,x)\rvert\). 
For instance:

\[
\alpha(x,A)
=
\begin{cases}
\alpha_{\mathrm{low}}(x), & \text{if } \lvert A\rvert \le A_{\mathrm{crit}},\\
\alpha_{\mathrm{high}}(x), & \text{if } \lvert A\rvert > A_{\mathrm{crit}},
\end{cases}
\quad
s(x,A)
=
\begin{cases}
s_{\mathrm{low}}(x), & \text{if } \lvert A\rvert \le A_{\mathrm{crit}},\\
s_{\mathrm{high}}(x), & \text{if } \lvert A\rvert > A_{\mathrm{crit}},
\end{cases}
\]
and similarly for
\[
F(A)
=
\begin{cases}
F_{\mathrm{low}}(A), & \lvert A\rvert \le A_{\mathrm{crit}},\\
F_{\mathrm{high}}(A), & \lvert A\rvert > A_{\mathrm{crit}}.
\end{cases}
\]

\begin{remark}[Amplitude Threshold in Single Domain]
In the single-domain setting (Parts~A--B), “amplitude” means \(\lvert A(t,x)\rvert\) 
for each point \(x\). Once \(\max_{x\in \Omega}\lvert A(t,x)\rvert\) exceeds 
\(A_{\mathrm{crit}}\), the PDE “switches” to the high regime. 
This threshold crossing can happen at a specific time \(t_b\). We define
\[
t_b
=
\inf\Bigl\{\,t>0:\,\max_{x\in\Omega}\lvert A(t,x)\rvert 
= A_{\mathrm{crit}}\Bigr\}.
\]
If $t_b=\infty$, we never cross the threshold. If $t_b<\infty$, at $t_b$ we switch 
exponents and possibly reinitialize memory (see below).
\end{remark}

\subsubsection{Memory Reinitialization vs.\ Global Memory}
\label{subsubsec:memory_reinit}

When \(\alpha\) or \(s\) changes abruptly at \(t_b\), we must clarify whether the 
Caputo derivative $\partial_t^{\alpha_{\mathrm{high}}}$ (for $t>t_b$) integrates over 
$\tau\in [0,t]$ (retaining \emph{global} memory) or $\tau\in[t_b,t]$ (reinitialized memory).  
Both are common in biological or physical systems (cf.\ \cite[Ch.~6]{Magin2006}, 
\cite{Sun2019}). Our well-posedness approach accommodates either scenario by treating 
$[0,t_b]\cup [t_b,T]$ as consecutive intervals with a \emph{fresh PDE definition} 
on $[t_b,T]$ if memory restarts, or else continuing the same convolution from $\tau=0$ 
if memory is global.

\subsubsection{Blowup, Saturation, or “Lumps”}
\label{subsubsec:blowup_saturate}

If the “high” exponents $\alpha_{\mathrm{high}}, s_{\mathrm{high}}$ and forcing 
$F_{\mathrm{high}}$ allow \emph{unbounded} amplitude growth, a finite-time blowup can 
occur, after which classical solutions fail. Alternatively, $F_{\mathrm{high}}$ might 
\emph{saturate} the amplitude, forming lumps at $\lvert A\rvert = A_{\mathrm{crit}}$.  
Our piecewise PDE arguments yield \emph{maximal‐time solutions} in each regime.  
We keep in mind the possibility that amplitude \emph{never} returns below $A_{\mathrm{crit}}$ 
once it crosses, or it might re-cross if $F_{\mathrm{high}}$ shrinks $A$ again.

\begin{remark}[Local vs. Global PDE Switching]
If $\alpha$ or $s$ also depends on the spatial coordinate $x$, then different regions 
in $\Omega$ can be in “low” or “high” regimes simultaneously. In principle, this yields 
a PDE with \(\alpha(x,A)\) and $s(x,A)$ that vary across $x\in \Omega$ depending on 
the local amplitude $A(t,x)$. The solution may develop localized lumps or blowups in 
subregions of $\Omega$.
\end{remark}

\subsubsection{Notation Summary for Part~C (ERM, Blowups)}

\begin{itemize}
\item $\Omega\subset\mathbb{R}^n$: bounded domain with Dirichlet boundary. 
\item $(-\Delta)^s$: spectral fractional Laplacian as in Parts~A--B, 
      $0<s<1$, $H_0^s(\Omega)$ the domain. 
\item $\alpha,\ s,\ F$ possibly \emph{piecewise} definitions: 
      $(\alpha_{\mathrm{low}}, s_{\mathrm{low}}, F_{\mathrm{low}})$ vs.\ 
      $(\alpha_{\mathrm{high}}, s_{\mathrm{high}}, F_{\mathrm{high}})$, 
      triggered by $\lvert A\rvert>A_{\mathrm{crit}}$. 
\item Memory reinitialization: on subintervals $[0,t_b]$ and $[t_b,T]$, 
      or global memory continuing from $\tau=0$. 
\item Blowup or lumps: if $F_{\mathrm{high}}$ is unbounded, amplitude 
      can blow up in finite time; if saturating, lumps remain finite.
\end{itemize}

These preliminary ideas set the stage for a \emph{piecewise in time} PDE solution, 
where we re-solve the PDE each time the amplitude crosses the threshold 
$A_{\mathrm{crit}}$.  The next subsections present the \textbf{well-posedness} and 
\textbf{blowup} analysis under amplitude-dependent exponents or forcing.
%---------------------------------------------------------
\subsection{Amplitude‐Based PDE Switching}
\label{subsec:amplitude_switching}
%---------------------------------------------------------

Consider a PDE of the form
\[
\partial_t^{\alpha(x,A)} A(t,x)
\;+\;
(-\Delta)^{s(x,A)} A(t,x)
\;=\;
F\bigl(A(t,x)\bigr),
\quad
A(0,x)=A_0(x),
\quad
A|_{\partial\Omega}=0,
\]
where the exponents \(\alpha(x,A)\) and \(s(x,A)\) (and possibly the forcing \(F(A)\))
are defined \emph{piecewise} depending on whether \(\lvert A\rvert \le A_{\mathrm{crit}}\)
or \(\lvert A\rvert>A_{\mathrm{crit}}\).  A schematic example is:

\[
\alpha(x,A) 
=
\begin{cases}
\alpha_{\mathrm{low}}(x),
  & \text{if }\lvert A\rvert \le A_{\mathrm{crit}}, 
\\[4pt]
\alpha_{\mathrm{high}}(x),
  & \text{if }\lvert A\rvert > A_{\mathrm{crit}},
\end{cases}
\quad
s(x,A)
=
\begin{cases}
s_{\mathrm{low}}(x), 
  & \text{if }\lvert A\rvert \le A_{\mathrm{crit}}, 
\\[4pt]
s_{\mathrm{high}}(x), 
  & \text{if }\lvert A\rvert > A_{\mathrm{crit}},
\end{cases}
\]
\[
F(A)
=
\begin{cases}
F_{\mathrm{low}}(A),
  & \text{if }\lvert A\rvert \le A_{\mathrm{crit}}, 
\\[4pt]
F_{\mathrm{high}}(A),
  & \text{if }\lvert A\rvert > A_{\mathrm{crit}}.
\end{cases}
\]
Intuitively, the system remains in the “low” PDE regime as long as
\(\max_{x\in\Omega}\lvert A(t,x)\rvert < A_{\mathrm{crit}}\).  If that amplitude threshold
is never crossed, a standard (fixed or variable) fractional PDE argument (see
Parts~A--B) ensures a unique global solution.  

\paragraph{Initial Condition \(\max|A_0|\le A_{\mathrm{crit}}\).}
If the initial data \(A_0\) also satisfies \(\max_{x}\lvert A_0(x)\rvert \le A_{\mathrm{crit}}\),
then we begin entirely in the “low” regime.  A unique solution continues in that regime as
long as amplitude \(\lvert A\rvert\) does not exceed \(A_{\mathrm{crit}}\).  If amplitude
exceeds \(A_{\mathrm{crit}}\) for some time \(t\), we switch PDE definitions.

%---------------------------------------------------------
\subsection{Threshold Time and Post‐Switch PDE}
\label{subsec:threshold_time}
%---------------------------------------------------------

Define the \textbf{first crossing time}
\[
t_b
\;=\;
\inf\Bigl\{
  t>0:\,
  \max_{x\in\Omega}\bigl\lvert A(t,x)\bigr\rvert
  \;=\;
  A_{\mathrm{crit}}
\Bigr\}.
\]
If \(t_b = \infty\), amplitude never reaches \(A_{\mathrm{crit}}\).  If \(t_b<\infty\),
at \(t_b\) we “switch” from
\((\alpha_{\mathrm{low}}, s_{\mathrm{low}}, F_{\mathrm{low}})\)
to
\((\alpha_{\mathrm{high}}, s_{\mathrm{high}}, F_{\mathrm{high}})\).
We impose continuity at \(t_b\):
\[
A\bigl(t_b^+\bigr)
\;=\;
\lim_{\tau\to t_b^-} A(\tau).
\]
A new PDE with “high” exponents is solved for \(t>t_b\), with initial data \(A(t_b)\).
This is exactly the piecewise approach from variable \(\alpha(t)\) but triggered by
\(\lvert A\rvert\) crossing a threshold.

\paragraph{Memory Across Threshold.}
A key modeling choice is whether the new Caputo derivative
\(\partial_t^{\alpha_{\mathrm{high}}}\) integrates over \(\tau\in[0,t]\) (global memory)
or \(\tau\in[t_b,t]\) (reinitialized memory).  Both scenarios appear in the literature
(\cite[Ch.\,6]{Magin2006}, \cite{Sun2019}).  From a well-posedness viewpoint, one typically
splits the time interval \([0,t_b]\cup[t_b,T]\) and solves a PDE on each piece with
appropriate initial data at \(t_b\).  \emph{Uniqueness} still holds provided \(F\) is
Lipschitz or monotone on each amplitude range.

\paragraph{Multiple Threshold Crossings.}
One can similarly allow amplitude to cross \(\lvert A\rvert=A_{\mathrm{crit}}\) multiple
times, toggling between “low” and “high” PDE definitions.  Each crossing subdivides
\([0,T]\) into intervals on which \(\alpha\) and \(s\) remain fixed.  The same piecewise
construction with continuity at each crossing point yields a solution, though regularity
might degrade if \(\alpha,\beta,F\) jump abruptly many times.

%---------------------------------------------------------
\subsection{Blowup vs.\ Lumps vs.\ Distributional Solutions}
\label{subsec:blowup_discussion}
%---------------------------------------------------------

If \((\alpha_{\mathrm{high}}, s_{\mathrm{high}}, F_{\mathrm{high}})\) truly allows
unbounded growth (e.g.\ superlinear \(F_{\mathrm{high}}\)), then amplitude may \emph{blow
up} in finite time.  In that case, the classical or weak solution can only continue up to
this \emph{blowup time} \(t_{\mathrm{bu}} < \infty\).  After blowup, one might adopt a
\emph{distributional} or \emph{measure‐valued} extension, or else treat blowup as a
physical “end” of the classical regime.  Alternatively, if \(F_{\mathrm{high}}\) saturates
amplitude, we obtain stable lumps or finite spikes at \(\lvert A\rvert = A_{\mathrm{crit}}\)
but do not blow up to infinity.

\begin{theorem}[SRFT + Blowup Well-Posedness, Piecewise]
\label{thm:ERMBlowup_expanded}
Let
\(\bigl(\alpha_{\mathrm{low}},\alpha_{\mathrm{high}},
       s_{\mathrm{low}},s_{\mathrm{high}},
       F_{\mathrm{low}},F_{\mathrm{high}}\bigr)\)
be piecewise definitions for the fractional exponents and forcing, with continuity at
\(\lvert A\rvert=A_{\mathrm{crit}}\). Then:

\begin{itemize}
\item \textbf{No blowup:} If the amplitude \(\max_x\lvert A(t,x)\rvert\) never reaches
      \(A_{\mathrm{crit}}\), the system remains in the “low” regime for all time, and
      a unique solution exists globally.  
\item \textbf{Threshold crossing:} If at \(t_b\), amplitude first hits \(A_{\mathrm{crit}}\),
      then for \(t>t_b\) we switch PDE definitions to
      \(\alpha_{\mathrm{high}}, s_{\mathrm{high}}, F_{\mathrm{high}}\).  Imposing
      \(A(t_b^+)=A(t_b^-)\) ensures continuity. If
      \(F_{\mathrm{high}}\) is Lipschitz or monotone, a unique local solution persists
      until a next threshold or blowup time.
\item \textbf{Blowup or lumps:} If \(F_{\mathrm{high}}\) induces unbounded growth,
      amplitude can blow up in finite time. We typically obtain a maximal‐time classical
      solution.  If, however, \(\lvert A\rvert\approx A_{\mathrm{crit}}\) saturates or
      lumps, we remain finite.  One can continue the PDE in the “high” regime
      indefinitely if no further blowups occur.
\end{itemize}
\end{theorem}

Moreover, in each sub-interval of time (and amplitude regime) 
where \(\alpha_{\mathrm{low}}\) or \(\alpha_{\mathrm{high}}\) remain fixed, 
the solution 
\[
A \,\in\, L^\infty\bigl(0,T;H_0^s(\Omega)\bigr)
\]
satisfies the PDE in the weak sense with boundary condition 
\(A|_{\partial\Omega}=0\) and matching continuity at threshold times. 
When \(F_{\mathrm{high}}\) is Lipschitz or monotone, uniqueness again follows 
by a fractional Grönwall estimate on the difference of solutions.

\paragraph{Remark on Non‐Lipschitz \(F_{\mathrm{high}}\).}
If \(F_{\mathrm{high}}\) is strongly nonlinear (e.g.\ superlinear growth without monotonicity),
uniqueness may fail. One‐sided Lipschitz or monotonicity can sometimes preserve uniqueness
(\cite[Ch.\,7]{Diethelm2010}), but otherwise one generally obtains only existence.  

Hence, \textbf{amplitude-triggered threshold logic} merges naturally with the previous
fixed/variable fractional PDE theory: each PDE regime is solved in a piecewise manner over
the time intervals where \(\lvert A\rvert\) remains below or above \(A_{\mathrm{crit}}\).
Memory can be reinitialized or continued at crossing times. Blowups can occur if the
“high” PDE regime is truly unbounded, or else amplitude saturates at lumps or plateaus
if forcing is limited. This completes the single-domain SRFT perspective on \emph{amplitude
blowups or lumps} for time-fractional PDEs.

\subsection{Beyond Blowup: Measure--Valued or Distributional Solutions}
\label{subsec:post_blowup_measure_valued}

Once the amplitude becomes unbounded at a finite time \(t_{\mathrm{bu}}<T\),
our classical (or weak) solution in \(L^\infty\bigl(0,T;H_0^s(\Omega)\bigr)\) 
ceases to exist beyond \(t_{\mathrm{bu}}\). In principle, one can define 
\emph{measure-valued} or \emph{distributional} solutions to continue 
the evolution past blowup; see, for instance, 
\cite[Section~4]{Zacher2005} or related references on fractional PDEs 
with weaker solution concepts. 

Such frameworks allow capturing post-blowup behaviors (e.g.\ mass concentration 
in measure form), but lie beyond the scope of this monograph. Our main focus 
is the classical regime where a unique solution exists as long as no finite-time 
blowup occurs.

\subsection*{From Single-Domain to Extended-Domain: A Preview}
\label{subsec:bridge_single_to_extended}

We have now established how fractional PDEs in a \textit{single bounded domain}~\(\Omega\) 
can exhibit:
\begin{itemize}
    \item \textbf{Well-posedness} under Caputo fractional derivatives (fixed or variable order),
    \item \textbf{Amplitude-triggered blowups or lumps} once \( |A|\) crosses \(A_{\mathrm{crit}}\),
    \item \textbf{Memory reinitialization or global memory} choices at threshold times.
\end{itemize}
In the next part, \textbf{Part~D}, we will \emph{extend} these ideas by replacing \(\Omega\) 
with a \textbf{higher-dimensional manifold} \( \mathcal{M} = \Omega \times \mathcal{A}\).  
The \emph{same} Galerkin and fractional Grönwall techniques carry over, but now wave 
interference in \(x\in\Omega\) and amplitude thresholds in \(a\in\mathcal{A}\) coexist 
in a single PDE.  This brings wave-like behavior, amplitude triggers, and blowups under 
one unified manifold framework, setting the stage for multi-dimensional phenomena.


%=========================================================
%=========================================================
\section{Part D: Extended Manifold PDE in \texorpdfstring{\(\Omega\times\mathcal{A}\)}{}}
\label{sec:intro}
%=========================================================
\addcontentsline{toc}{section}{Introduction to Part D}
\label{sec:intro_partD}
%=========================================================

In this final part, we move beyond the single-domain setting \(\Omega\subset\mathbb{R}^n\) 
and introduce the \emph{extended manifold}
\[
   \mathcal{M} \;=\; \Omega \times \mathcal{A}.
\]
Here, \(\Omega\) is still our physical domain, but \(\mathcal{A}\) now serves as a 
\emph{wave-forcing} or \emph{amplitude} dimension, allowing us to embed additional 
variables or forcing laws directly into the PDE’s spatial coordinates. 
Despite working in a bigger domain, our \emph{core methodology}---namely \textbf{Galerkin 
projections}, \textbf{fractional Grönwall estimates}, and \textbf{piecewise PDE definitions 
for amplitude triggers}---remains exactly the same.  Indeed, all arguments from Parts~A--C 
carry over once we replace \(x\in\Omega\) by \(\mathbf{z}=(x,a)\in \mathcal{M}\).

Moreover, \textbf{amplitude thresholds} can still occur at \(\mathbf{z}=(x,a)\) 
whenever \( |U(t,x,a)|\) crosses a critical level \(A_{\mathrm{crit}}\).  In the 
\(\mathcal{M}\)-setting, one might interpret these as \emph{lumps in the amplitude 
coordinates} (if \(\mathcal{A}\) is an amplitude axis), or \emph{threshold logic in a 
multi-dimensional manifold} if \(\mathcal{A}\) encodes additional parameters.  Thus, 
the single-domain threshold approach generalizes naturally to a higher-dimensional 
wave-amplitude PDE framework.

\bigskip

\noindent
\textbf{Overview of Part~D.} Below, we will:
\begin{itemize}
    \item Define the \emph{spectral fractional Laplacian} on \(\mathcal{M}\), 
          ensuring boundary or decay conditions in both \(\Omega\) and \(\mathcal{A}\).
    \item Formulate the extended PDE with wave forcing and/or amplitude dimension, 
          showing how amplitude triggers can still be enforced piecewise in time.
    \item Repeat the same \(\bigl(\!\)\emph{Galerkin + fractional Grönwall}\(\bigr)\) 
          arguments to show existence, uniqueness, and blowup or lump scenarios in 
          \(\mathcal{M}\).
\end{itemize}
Thus, we confirm that \textbf{the bigger domain does not change} the fundamental approach, 
but allows for a richer interplay of wave interference and amplitude threshold phenomena.


In \textbf{Part~D}, we move beyond the single-domain setting of $\Omega\subset\mathbb{R}^n$ 
and embed our fractional PDE into a \emph{higher-dimensional} manifold 
\(\mathcal{M}=\Omega\times \mathcal{A}\). Here, the extra coordinate \(a\in\mathcal{A}\) 
can represent an amplitude dimension, a probability variable, or any auxiliary coordinate 
that enriches the PDE with additional physical or interpretative layers. 

\paragraph{Motivation and Context.}
In many physical and engineering scenarios, wave interference, amplitude thresholds, or 
probability-like amplitudes arise in a natural way. By formulating a \emph{single PDE} 
on \(\Omega\times \mathcal{A}\), we unify:
\begin{itemize}
  \item \textbf{Wave/field evolution} in the physical coordinates \(x\in\Omega\),
  \item \textbf{Amplitude- or probability-like dynamics} in the extended dimension 
        \(a\in\mathcal{A}\),
  \item \textbf{Fractional memory and threshold switching} across both spatial and 
        amplitude directions. 
\end{itemize}
This approach can accommodate amplitude-triggered blowups (as in Part~C) and wave forcing 
in a single manifold PDE, thereby providing a more complete picture of multi-scale, 
nonlocal phenomena.

\paragraph{Key Themes and Methodology.}
\begin{itemize}
  \item \emph{Spectral Fractional Laplacian on \(\mathcal{M}\)}: 
    We define the fractional operator $(-\Delta_{\mathbf{z}})^s$ on the product domain 
    \(\mathbf{z}=(x,a)\in \mathcal{M}\). A new eigenfunction basis \(\{\Phi_k\}\) 
    arises, each corresponding to eigenvalues \(\lambda_k^s\).
  \item \emph{Extended Galerkin Approximation}: We expand \(U(t,\mathbf{z})\) in the 
    basis \(\{\Phi_k\}\), leading again to a system of Caputo ODEs in time, but now 
    capturing wave or amplitude interactions in higher dimensions.
  \item \emph{Uniform Bounds \& Fractional Gr\"onwall}: We still rely on the fractional 
    product rule and Gr\"onwall-type estimates (possibly adapted to $\alpha(\cdot)$ 
    or amplitude triggers) to ensure $U$ remains bounded in 
    $L^\infty(0,T;H_0^s(\mathcal{M}))$.
  \item \emph{Blowup or Saturation in \(\Omega\times \mathcal{A}\)}: 
    If a threshold $\lvert U\rvert=A_{\mathrm{crit}}$ is encountered, or if forcing 
    allows unbounded amplitude, local lumps or blowups can form in subregions of 
    $\mathcal{M}$. This unifies amplitude-threshold logic (Part~C) with wave interference 
    or harmonic forcing in $x$.
\end{itemize}

\paragraph{Relation to Parts~A--C.}
\begin{itemize}
    \item The core \textbf{Galerkin + fractional Grönwall} argument from the single-domain 
          setting extends naturally to $\mathcal{M}=\Omega\times \mathcal{A}$. 
          We merely replace the $\{\phi_k\}$ eigenbasis in $\Omega$ with a 
          $\{\Phi_k\}$ eigenbasis in $\mathcal{M}$.
    \item Amplitude triggers and blowups, introduced in Part~C, can be interpreted now 
          as \emph{local} threshold crossings in $(x,a)$-space, generating piecewise PDE 
          definitions. 
    \item Wave or \emph{probability-like} dimensions appear seamlessly, capturing 
          interference patterns or probability amplitudes in a single PDE framework.
\end{itemize}

\paragraph{Outline of Part~D.}
\begin{itemize}
  \item \emph{Preliminaries and Notation}: We define the product manifold 
        \(\mathcal{M}=\Omega\times \mathcal{A}\), specify boundary or decay conditions 
        on \(\partial\mathcal{M}\), and set up $H_0^s(\mathcal{M})$.
  \item \emph{Formulating the Extended PDE}: 
        We present a single PDE of the form 
        \[
          \partial_t^{\alpha(\cdots)} U(t,\mathbf{z})
          +
          \mathcal{L}_{\mathbf{z}}^{s(\cdots)} U(t,\mathbf{z})
          =
          \mathcal{F}\bigl(t,\mathbf{z}, U,\nabla_{\mathbf{z}}U\bigr),
        \]
        possibly with wave forcing or amplitude-based blowups. 
  \item \emph{Well-Posedness}: 
        By applying a higher-dimensional Galerkin scheme, 
        fractional product rules, and fractional Aubin--Lions--Simon arguments, we show 
        existence and uniqueness in $L^\infty(0,T;H_0^s(\mathcal{M}))$, piecewise in time 
        if amplitude triggers occur.
  \item \emph{Interpretations and Next Steps}: 
        We highlight how wave forcing leads to interference patterns in $x$, while 
        amplitude thresholds produce discrete lumps or blowups in $a$. This can be seen 
        as an SRFT viewpoint bridging continuous waves and discrete amplitude events.
\end{itemize}

\medskip

In what follows, we first set up the \textbf{Preliminaries and Notation} for the 
extended manifold, describing how to define $(-\Delta_{\mathbf{z}})^s$ with boundary or 
decay conditions. We then proceed with the well-posedness proof, mirroring the single-domain 
analysis but in higher dimensions, ultimately showing that amplitude triggers and memory 
reinitialization extend naturally to $\Omega\times\mathcal{A}$.

%---------------------------------------------------------------------------
\subsection{Preliminaries and Notation (Part D: Extended Manifold PDE)}
\label{subsec:prelim_partD}
%---------------------------------------------------------------------------

In \textbf{Part~D}, we generalize the single-domain approach (Parts~A--C) to a higher-dimensional 
manifold 
\[
\mathcal{M} \;=\; \Omega \times \mathcal{A},
\]
where \(\Omega\subset\mathbb{R}^n\) is the usual physical domain, and 
\(\mathcal{A}\subseteq\mathbb{R}^m\) (or a suitable manifold) represents an amplitude 
or probability coordinate. This subsection sets up the extended manifold, boundary/decay 
conditions, and the **fractional operators** in \(\mathcal{M}\). We also define the 
associated Sobolev-like space \(H_0^s(\mathcal{M})\) and recall how **Caputo derivatives** 
in time couple with amplitude or wave forcing in \(\mathbf{z}=(x,a)\).

\subsubsection{Domain \texorpdfstring{\(\mathcal{M} = \Omega\times \mathcal{A}\)}{} and Boundary/Decay Conditions}
\label{subsubsec:domain_M}

Let \(\Omega\subset\mathbb{R}^n\) be a bounded domain with a sufficiently regular boundary 
(e.g.\ Lipschitz or $C^{1,\alpha}$). Let \(\mathcal{A}\subseteq \mathbb{R}^m\) (or more 
generally a manifold) stand for an amplitude or probability dimension. Then the 
\emph{extended manifold} is
\[
\mathcal{M}
\;=\;
\Omega \,\times\, \mathcal{A}.
\]
We denote a typical point by \(\mathbf{z}=(x,a)\), with $x\in\Omega$ and $a\in\mathcal{A}$.

\paragraph{Boundary of \(\mathcal{M}\).}
We write
\[
\partial\mathcal{M}
\;=\;
(\partial\Omega \,\times\, \mathcal{A})
\;\cup\;
(\Omega \,\times\, \partial\mathcal{A}).
\]
Depending on whether $\mathcal{A}$ is bounded or unbounded, we impose either 
\emph{Dirichlet boundary} ($U=0$ on $\partial\mathcal{A}$) or \emph{decay at infinity} 
if $a\in\mathbb{R}^m$ unbounded. Combining this with $U|_{\partial \Omega}=0$ yields 
the global boundary condition $U=0$ on $\partial\mathcal{M}$, or appropriate decay if 
\(\lvert a\rvert\to \infty\). These conditions ensure a well-defined fractional operator 
on $\mathcal{M}$.

\subsubsection{Fractional Sobolev Space \texorpdfstring{$H_0^s(\mathcal{M})$}{}}
\label{subsubsec:H0sM}

We define the \emph{spectral fractional Laplacian} on $\mathcal{M}$ in the same spirit as 
Parts~A--C but extended to the product domain. Let $\{\Phi_k\}$ be the Dirichlet 
eigenfunctions of $-\Delta_{\mathbf{z}}$ on $\mathcal{M}$, i.e.\
\[
-\Delta_{\mathbf{z}} \,\Phi_k(\mathbf{z})
=
\lambda_k \,\Phi_k(\mathbf{z}),
\quad
\Phi_k\big|_{\partial\mathcal{M}}=0,
\quad
\langle \Phi_j,\Phi_k\rangle_{L^2(\mathcal{M})}
=
\delta_{jk}.
\]
Then, for $s\in(0,1)$, the \emph{fractional} operator $(-\Delta_{\mathbf{z}})^s$ is given by
\[
(-\Delta_{\mathbf{z}})^s U
=
\sum_{k=1}^\infty
  \lambda_k^s \,\widehat{U}_k \,\Phi_k(\mathbf{z}),
\quad
\widehat{U}_k
=
\int_{\mathcal{M}}
  U(\mathbf{z})\,\Phi_k(\mathbf{z})
\,d\mathbf{z}.
\]
The corresponding \emph{domain} is $H_0^s(\mathcal{M})$, with norm
\[
\|U\|_{H_0^s(\mathcal{M})}^2
=
\sum_{k=1}^\infty
  \lambda_k^s \,\lvert \widehat{U}_k\rvert^2,
\]
plus an $L^2(\mathcal{M})$ tSRFT if desired (equivalent up to constants). Under Dirichlet 
or suitable decay boundary conditions, we interpret $U|_{\partial\mathcal{M}}=0$ in the 
fractional-trace sense. See \cite{DiNezzaPalatucciValdinoci} or \cite[Ch.~1]{Kilbas2006} 
for further comparisons with the integral definition.

\subsubsection{Caputo Derivatives in Time with \texorpdfstring{\(\alpha(\cdot,\mathbf{z},U)\)}{}}
\label{subsubsec:caputo_time_M}

We still employ a \emph{Caputo} fractionald derivative in $t\in[0,T]$, now possibly with 
\(\alpha=\alpha(t,\mathbf{z},U)\) referencing amplitude or wave triggers. That is,
\[
\partial_t^{\alpha(\cdot)} U(t,\mathbf{z})
=
\frac{1}{\Gamma\bigl(1-\alpha(t,\mathbf{z},U)\bigr)}
\int_0^t
  \frac{\partial_\tau U(\tau,\mathbf{z})}{
         (t-\tau)^{\alpha(t,\mathbf{z},U)}}
\,d\tau,
\]
interpreted piecewise if $\alpha$ changes across threshold events. 
We require $U$ to be sufficiently smooth in time (or piecewise $C^1$) so that 
this convolution integral is well-defined. As in previous parts, \emph{memory 
reinitialization vs.\ global memory} upon crossing thresholds or amplitude lumps 
remains a modeling choice (see \cite{Magin2006,Sun2019}).

\subsubsection{Notation Summary for Part~D}

\begin{itemize}
\item $\mathcal{M} = \Omega \times \mathcal{A}$: extended manifold with coordinates 
      $\mathbf{z}=(x,a)\in \mathcal{M}$; boundary or decay at 
      $\partial\mathcal{M}=(\partial\Omega\times\mathcal{A})\cup (\Omega\times \partial\mathcal{A})$.
\item $H_0^s(\mathcal{M})$: spectral fractional Sobolev space of order $s\in(0,1)$, 
      with norm $\|U\|_{H_0^s(\mathcal{M})}^2 = \sum \lambda_k^s \,\lvert \widehat{U}_k\rvert^2$.
\item $\partial_t^{\alpha(\cdot)}$: Caputo time derivative, possibly variable in 
      $(t,\mathbf{z},U)$ or triggered by amplitude threshold, piecewise interpreted if 
      $\alpha$ changes. 
\item $\mathcal{F}\bigl(t,\mathbf{z},U,\nabla_{\mathbf{z}} U\bigr)$: a \emph{nonlinear forcing} 
      tSRFT that can incorporate wave or harmonic influences (e.g.\ $\sin(\kappa\cdot x-\omega t)$), 
      amplitude saturation, or memory integrals. 
\end{itemize}

Hence, in \textbf{Part~D}, we will pose a \emph{single PDE} on $\mathcal{M}\times [0,T]$, 
e.g.
\[
\partial_t^{\alpha(\cdot)} U(t,\mathbf{z})
+
(-\Delta_{\mathbf{z}})^s U(t,\mathbf{z})
=
\mathcal{F}\bigl(t,\mathbf{z},U,\nabla_{\mathbf{z}} U\bigr),
\quad
U(0,\mathbf{z})=U_0(\mathbf{z}),
\quad
U\big|_{\partial\mathcal{M}}=0\text{ (or decay)},
\]
and apply the \textbf{Galerkin + fractional Gr\"onwall} strategy in higher dimensions. 
We must carefully manage boundary or decay at $\lvert a\rvert\to \infty$ if $\mathcal{A}$ 
is unbounded, and piecewise definitions of $\alpha,\ s,\ F$ if amplitude triggers occur. 
The next sections detail these steps, paralleling the single-domain methods from Parts~A--C 
but now in $\Omega\times \mathcal{A}$.

%=========================================================
%=========================================================
\subsection{Definition of the Extended Manifold and PDE (Part D)}
\label{subsec:extended_manifold_setup}
%=========================================================

\subsubsection{Domain Setup and Extended Domain \texorpdfstring{\(\mathcal{M}\)}{}}

\begin{definition}[Extended Domain \(\mathcal{M}\)]
\label{def:extended_domain}
Let \(\Omega \subset \mathbb{R}^n\) be a bounded domain with boundary \(\partial \Omega\). 
We assume \(\Omega\) is sufficiently regular (e.g.\ Lipschitz or \(C^{1,\alpha}\)) so that 
standard fractional Sobolev space theory applies.

Let \(\mathcal{A} \subset \mathbb{R}^m\) (or a suitable manifold) represent an ``amplitude'' 
or ``probability'' coordinate space. We consider common scenarios for \(\mathcal{A}\), e.g.:
\begin{enumerate}
    \item \(\mathcal{A} = \mathbb{R}\), for a full unbounded amplitude range,
    \item \(\mathcal{A} = [0,\infty)\), if modeling a nonnegative amplitude/probability dimension,
    \item \(\mathcal{A}\) is a \emph{bounded} set (e.g.\ \([a_{\min}, a_{\max}]\)) to ensure finite measure and simpler boundary conditions.
\end{enumerate}
In all cases, define
\[
\mathcal{M} \;=\; \Omega \,\times\, \mathcal{A},
\]
and denote points in \(\mathcal{M}\) by \(\mathbf{z} = (x,a)\), where \(x \in \Omega\) 
and \(a \in \mathcal{A}\).
\end{definition}

\paragraph{Boundary of \(\mathcal{M}\).}
The boundary \(\partial \mathcal{M}\) decomposes as
\[
\partial \mathcal{M}
\;=\;
(\partial \Omega \,\times\, \mathcal{A})
\;\cup\;
(\Omega \,\times\, \partial \mathcal{A}),
\]
where \(\partial \Omega\) is the boundary in the physical coordinates, and 
\(\partial \mathcal{A}\) is the boundary (or limit) in the amplitude space. Thus,
\[
\mathbf{z} \in \partial \mathcal{M}
\;\Longleftrightarrow\;
\bigl(x \in \partial \Omega,\, a \in \mathcal{A}\bigr)
\quad\text{or}\quad
\bigl(x \in \Omega,\, a \in \partial \mathcal{A}\bigr).
\]
Depending on whether \(\mathcal{A}\) is bounded or unbounded:

\begin{itemize}
    \item If \(\mathcal{A}\) is \emph{bounded} (e.g.\ a closed interval \([a_{\min}, a_{\max}]\)), 
          then \(\partial \mathcal{A}\) are the endpoints.  
    \item If \(\mathcal{A}\) is unbounded (like \(\mathbb{R}\) or \([0,\infty)\)), we typically define 
          \(\partial \mathcal{A} \equiv \{\pm\infty\}\) in a formal sense and impose \emph{decay} as 
          \(\lvert a\rvert \to \infty\) (or \(a \to \infty\)) rather than a strict boundary condition.
\end{itemize}

\paragraph{Boundary/Decay Conditions on $U$.}
To ensure the fractional Laplacian (or other fractional operators) on \(\mathcal{M}\) 
is well-defined, we impose one of the following on \(U(t,\mathbf{z})\):

\begin{enumerate}
\item \emph{Dirichlet condition}:
  \[
     U(t,\mathbf{z})\big|_{\mathbf{z}\in \partial \mathcal{M}} = 0,
     \quad
     t \in [0,T].
  \]
  This applies if \(\partial \Omega\) and \(\partial \mathcal{A}\) are genuine boundaries 
  and we want the ``Dirichlet fractional Laplacian'' in \(\mathbf{z}\).

\item \emph{Decay at infinity}:
  If \(\mathcal{A}\) is unbounded, we assume
  \[
    \lim_{\lvert a\rvert \to \infty} U(t,x,a) = 0
    \quad\text{(uniformly in $x$ and $t$)},
  \]
  and still set \(U(t,x,a)=0\) on \(\partial\Omega\).  Such decay is common in fractional 
  PDE theory for unbounded domains \cite{Kilbas2006}, ensuring integrable tails in the 
  integral definition of the fractional Laplacian.

\item \emph{Periodic or reflective boundary}: 
  In certain models, \(\mathcal{A}\) might be periodic.  Then $U$ is $a$-periodic, 
  eliminating boundary terms in the $a$-direction. This is less common for an amplitude 
  dimension unless $a$ is truly cyclical.
\end{enumerate}

\begin{remark}[Geometric Regularity of \(\Omega\)]
We typically assume \(\Omega\) is Lipschitz (or $C^{1,\alpha}$) for the fractional 
Sobolev embeddings. If \(\Omega\) is highly irregular, one still defines fractional PDEs, 
but the well-posedness proofs demand more advanced measure-theoretic arguments.
\end{remark}

\begin{remark}[Measure of \(\mathcal{M}\)]
If \(\mathcal{A}\) is bounded, then \(\mathcal{M}\) has finite measure and $L^2(\mathcal{M})$ 
is standard.  If \(\mathcal{A}\) is unbounded, $\mathcal{M}$ might be infinite measure, so 
one requires integrability conditions for $a$-tails (see \cite{DiNezzaPalatucciValdinoci}).  The essential 
difference from $\Omega$ alone is the \emph{combined geometry} in $(x,a)$.
\end{remark}

\paragraph{Interpretation.}
\begin{itemize}
    \item $x \in \Omega$ are the usual spatial coordinates (e.g.\ $n$-dimensional),
    \item $a \in \mathcal{A}$ tracks amplitude or probability, forming an ``augmented'' space.  
    \item The boundary condition $U=0$ on $\partial \mathcal{M}$ can represent forced zero amplitude 
          in both physical and amplitude directions, or simplified decaying tails.
\end{itemize}
Hence, the domain setup in Definition~\ref{def:extended_domain} covers both bounded and 
unbounded amplitude coordinates, ensuring a rigorous fractional operator in $\mathcal{M}$.

%---------------------------------------------------------
\subsubsection{Proposed PDE on \texorpdfstring{\(\mathcal{M}\times [0,T]\)}{}: Expanded Details}
\label{subsubsec:extended_pde_formulation}
%---------------------------------------------------------

We now pose the higher-dimensional PDE that generalizes the SRFT framework:

\begin{equation}
\label{eq:extendedPDE_detailed}
\begin{cases}
\displaystyle
\partial_{t}^{\alpha\bigl(t,\mathbf{z},U\bigr)}\,U(t,\mathbf{z})
\;+\;
\mathcal{L}_{\mathbf{z}}^{\,s\bigl(t,\mathbf{z},U\bigr)}\, U(t,\mathbf{z})
\;=\;
\mathcal{F}\bigl(t,\mathbf{z}, U, \nabla_{\mathbf{z}}U\bigr),
& \quad t \in (0,T],\;\mathbf{z}\in \mathcal{M},
\\[6pt]
U(0,\mathbf{z}) = U_0(\mathbf{z}),
& \quad \mathbf{z}\in \mathcal{M},
\\[6pt]
U\big|_{\partial\mathcal{M}}=0 \text{ or decay conditions},
& \quad \mathbf{z}\in \partial\mathcal{M}.
\end{cases}
\end{equation}

Here:

\begin{itemize}
\item $\mathbf{z} = (x,a)\in\mathcal{M}$ merges the physical coordinate $x\in\Omega$ 
      with amplitude/probability $a\in \mathcal{A}$. 
\item $\partial_t^{\alpha(\cdot)}$ is a Caputo derivative in time, possibly variable in $t$ 
      or amplitude triggers, as in Parts~B--C.
\item $\mathcal{L}_{\mathbf{z}}^{s(\cdot)}$ is a fractional operator in $\mathbf{z}$, 
      typically $( -\Delta_{\mathbf{z}} )^s$ with exponent $s(\cdot)$ if amplitude 
      thresholds also alter $s$.
\item $\alpha$ and $s$ may switch upon amplitude crossing $|U|=A_{\mathrm{crit}}$, 
      continuing the SRFT amplitude-threshold logic.  
\item $\mathcal{F}\bigl(t,\mathbf{z},U,\nabla_{\mathbf{z}}U\bigr)$ might incorporate 
      wave forcing (e.g.\ $\sin(\kappa x-\omega t)\,U$), nonlinear saturations ($|U|^2 U$), 
      or memory integrals, all in the extended coordinate $\mathbf{z}$.
\item $U_0(\mathbf{z})\in H_0^s(\mathcal{M})$ is the initial data at $t=0$, consistent 
      with the boundary/decay conditions for $\mathbf{z}\in\partial\mathcal{M}$.
\end{itemize}

% Possibly an additional remark about wave interference or amplitude lumps.

\begin{remark}[Wave Interference and Probability Dimensions]
If $\mathcal{F}$ includes $\sin(\kappa \cdot x-\omega t)\,U$ or other wave terms, 
the physical coordinate $x$ can exhibit classical interference patterns. Meanwhile, 
the amplitude coordinate $a$ triggers threshold transitions or blowups. This allows 
a \emph{single PDE} to unify wave-like propagation in $x$ with amplitude-triggered 
blowups in $a$, echoing the SRFT viewpoint of bridging continuous wave phenomena 
and discrete amplitude events.
\end{remark}

\begin{remark}[Relation to Single-Domain Parts A--C]
Setting $m=0$ (so $\mathcal{A}$ is trivial), or ignoring amplitude thresholds, recovers 
the standard single-domain PDE. All well-posedness steps remain analogous, but 
\emph{dimensions} are higher in $\mathbf{z}$. The same Galerkin + fractional Grönwall 
framework applies \emph{a fortiori} in $\mathcal{M}=\Omega\times\mathcal{A}$.
\end{remark}

Hence, Part~D details how to carry over the single-domain fractional PDE arguments 
(Galerkin approximation, fractional product rule, Aubin--Lions--Simon compactness, 
etc.) to the extended manifold $\mathcal{M}$, accommodating wave interference, 
amplitude thresholds, and fractional memory in \emph{one} PDE.
%=========================================================
%=========================================================
\subsection{Functional Setting and Weak Formulation (Part D)}
\label{sec:functional-setting_D}
%=========================================================

In \textbf{Part~D}, we follow the same \emph{fractional Galerkin} and \emph{weak formulation} approach 
as in Parts~A--C but now on the \emph{extended manifold} 
\(\mathcal{M}=\Omega\times \mathcal{A}\). 
Rather than restate every detail from the single-domain setting, we briefly outline the 
function spaces and the weak formulation \emph{specific} to \(\mathcal{M}\). 
(See Parts~A and~B for the single-domain definitions, and Part~C for amplitude-triggered 
exponents.)

\subsubsection{Function Spaces on \texorpdfstring{\(\mathcal{M}\)}{}}
\label{subsubsec:function-spaces_M}

\paragraph{\texorpdfstring{$L^2(\mathcal{M})$}{L2(M)} and \texorpdfstring{$H_0^s(\mathcal{M})$}{H0s(M)}.}
We adopt the same definitions as in Parts~A--B, now in the product domain 
\(\mathcal{M}\subset \mathbb{R}^{n+m}\). 
\begin{itemize}
  \item \(\displaystyle L^2(\mathcal{M})\) consists of square-integrable functions 
        $U:\mathcal{M}\to\mathbb{R}$ with $\int_{\mathcal{M}} |U|^2<\infty$.  
  \item \(\displaystyle H_0^s(\mathcal{M})\), for $0<s<1$, is the domain of the \emph{spectral} 
        fractional Laplacian $(-\Delta_{\mathbf{z}})^s$ under Dirichlet or decay boundary 
        conditions on $\partial \mathcal{M}$ (cf.\ the single-domain definition in 
        Part~A, adapted to $\mathbf{z}\in\mathcal{M}$).  
\end{itemize}
If $\mathbf{z}\mapsto U(\mathbf{z})$ vanishes on $\partial\mathcal{M}$ (or decays at infinity 
if $\mathcal{A}$ is unbounded), we say $U\in H_0^s(\mathcal{M})$. 
The associated norm \(\|U\|_{H_0^s(\mathcal{M})}\) is defined via the \textbf{Dirichlet} 
eigenfunction expansion in $\mathcal{M}$ or via an integral formula with 
\(\bigl(U(\mathbf{z})-U(\mathbf{z}')\bigr)^2/\|\mathbf{z}-\mathbf{z}'\|^{n+m+2s}\). 

\begin{remark}[Reference to Single-Domain Sobolev Spaces]
All **technical details** (spectral vs.\ integral definition, boundary/decay conditions, 
equivalence of norms) are \emph{direct} analogs of the single-domain $H_0^s(\Omega)$ 
theory (see Part~A). The only difference is $\mathbf{z}\in\mathcal{M}$ and 
$\partial\mathcal{M}$ replaces $x\in\Omega$ and $\partial\Omega$.
\end{remark}

\subsubsection{Weak Form of the Extended PDE}
\label{subsubsec:weak_form_M}

Recall that \(\mathbf{z}=(x,a)\in \mathcal{M}\) and $t\in[0,T]$. We consider an 
\emph{extended PDE} of the form
\[
\partial_t^{\alpha(\cdots)} U(t,\mathbf{z})
\;+\;
\mathcal{L}_{\mathbf{z}}^{s(\cdots)} \,U(t,\mathbf{z})
\;=\;
\mathcal{F}\bigl(t,\mathbf{z},U,\nabla_{\mathbf{z}}U\bigr),
\quad
U(0,\mathbf{z})=U_0(\mathbf{z}),
\quad
U|_{\partial\mathcal{M}}=0 \text{ (or decay)},
\]
where $\partial_t^{\alpha(\cdots)}$ is a Caputo derivative in time (possibly with 
variable order or amplitude triggers), and 
$\mathcal{L}_{\mathbf{z}}^{s(\cdots)}$ is a fractional operator in $\mathbf{z}$. 
\emph{For the exact definitions}, see the single-domain analog in \S\ref{sec:partA_weak_form} 
(Part~A) and the amplitude switching logic in \S\ref{subsec:threshold_time} (Part~C).

\paragraph{Test Functions and Weak Solution.}
We multiply by a test function $\Phi\in H_0^s(\mathcal{M})$ and integrate over 
$\mathbf{z}\in \mathcal{M}$. As in Part~A’s weak formulation, we obtain
\[
\int_{\mathcal{M}}
  \Phi(\mathbf{z})\,\partial_t^{\alpha(\cdots)}U(t,\mathbf{z})
  \,d\mathbf{z}
\;+\;
\int_{\mathcal{M}}
  \Phi(\mathbf{z})\,\mathcal{L}_{\mathbf{z}}^{s(\cdots)}U(t,\mathbf{z})
  \,d\mathbf{z}
\;=\;
\int_{\mathcal{M}}
  \Phi(\mathbf{z})\,\mathcal{F}\bigl(t,\mathbf{z},U\bigr)\,d\mathbf{z}.
\]
The fractional product rule or Volterra kernel arguments from Part~A (or Part~B if $\alpha$ 
is variable in time) apply \emph{verbatim}, except $\Omega$ is replaced by $\mathcal{M}$. 
Boundary terms on $\partial\mathcal{M}$ vanish due to $U=\Phi=0$ (or decay).

\begin{definition}[Weak Solution on \(\mathcal{M}\)]
\label{def:weak_soln_extended}
A function 
\[
U \;\in\; L^\infty\bigl(0,T;\,H_0^s(\mathcal{M})\bigr)
\]
is called a \emph{weak solution} if, for each $t\in(0,T]$ and all test functions 
$\Phi\in H_0^s(\mathcal{M})$, the above integral identity holds. Additionally, $U(0,\cdot)=U_0$ 
and $U\big|_{\partial\mathcal{M}}=0$ (or satisfies the relevant decay condition if $\mathcal{A}$ 
is unbounded). We require $U$ to be sufficiently regular in time to define 
$\partial_t^{\alpha(\cdot)}U$ in a weak sense (cf.\ \cite{Diethelm2010,Kilbas2006}).
\end{definition}

\paragraph{No Repetition of Single-Domain Details.}
All further steps—Galerkin approximation, fractional energy estimates, passing to the limit, 
handling amplitude-triggered or wave forcing—follow exactly as in Parts~A--C, except in 
the extended coordinate \(\mathbf{z}\). We thus **omit** repeated derivations and 
refer to the single-domain proofs for the underlying Caputo ODE manipulations. 

\begin{remark}[Piecewise in Time if Amplitude Thresholds Occur]
If amplitude triggers cause $\alpha$ or $s$ to switch at $\lvert U\rvert=A_{\mathrm{crit}}$, 
then \S\ref{subsec:threshold_time} logic (Part~C) extends naturally: define sub-intervals in 
time around threshold crossing, re-solve the PDE with updated exponents on each sub-interval, 
and match at crossing times. The manifold dimension does not alter this piecewise approach.
\end{remark}

Hence, the **weak formulation** for the extended PDE \emph{mirrors} the single-domain 
case. We just note that the domain of integration and the fractional Laplacian are now 
in $\mathbf{z}\in\mathcal{M}$ rather than $x\in\Omega$.
%=========================================================
%=========================================================
\subsection{Galerkin Approximation in \texorpdfstring{\(\mathcal{M}\)}{M}}
\label{sec:galerkin_D}
%=========================================================

Here, we briefly outline how the \emph{Galerkin scheme} used in Parts~A--C extends to the 
\textbf{extended manifold} \(\mathcal{M}=\Omega\times \mathcal{A}\). Rather than restate 
all details, we only highlight the key differences from the single-domain version. 

\subsubsection{Spectral Decomposition in \(\mathcal{M}\)}

Let \(\{\Phi_k\}_{k=1}^\infty\subset H_0^s(\mathcal{M})\) be an orthonormal basis of 
\(L^2(\mathcal{M})\) satisfying 
\[
(-\Delta_{\mathbf{z}})^s \Phi_k(\mathbf{z}) 
= 
\lambda_k^s \,\Phi_k(\mathbf{z}),
\quad
\mathbf{z}\in \mathcal{M},
\quad
\Phi_k\big|_{\partial\mathcal{M}}=0,
\quad
\langle \Phi_j,\Phi_k\rangle_{L^2(\mathcal{M})}=\delta_{jk}.
\]
(See Part~A for the analogous single-domain construction and \cite{DiNezzaPalatucciValdinoci} 
for fractional Sobolev theory.) Here, \(\lambda_k^s\to\infty\) as \(k\to\infty\).

\subsubsection{Truncated Approximation and Projection}

We approximate the solution \(U(t,\mathbf{z})\) by 
\[
U_N(t,\mathbf{z})
=
\sum_{k=1}^N u_k^N(t)\,\Phi_k(\mathbf{z}),
\]
with initial coefficients $u_k^N(0)=\langle U_0,\Phi_k\rangle_{L^2(\mathcal{M})}$ to match 
\(U_0(\mathbf{z})\). Projecting the extended PDE onto each $\Phi_j$ yields a system of 
\emph{Caputo ODEs}, exactly as in Parts~A--C (except $\Omega$ is replaced by $\mathcal{M}$). 
Concretely, if 
\[
\partial_t^{\alpha(\cdots)} U + \mathcal{L}_{\mathbf{z}}^s U = \mathcal{F}(U),
\]
then, multiplying by $\Phi_j$ and integrating gives:
\[
\partial_t^{\alpha(\cdots)} u_j^N(t)
+ \lambda_j^s\,u_j^N(t)
=
R_j^N(t),
\]
where
\[
R_j^N(t)
= 
\int_{\mathcal{M}} \Phi_j(\mathbf{z})\, \mathcal{F}\Bigl(U_N(t,\mathbf{z})\Bigr)\,d\mathbf{z}.
\]
Standard Caputo ODE theory (cf.\ \cite[Ch.~2]{Diethelm2010} or Part~A) ensures a unique 
finite-dimensional solution $\mathbf{u}^N(t)=(u_1^N,\dots,u_N^N)$.

\subsubsection{Energy Estimates and Fractional Gr\"onwall}

Define the \textbf{energy} 
\[
E_N(t)
=
\|U_N(t)\|_{H_0^s(\mathcal{M})}^2
=
\sum_{j=1}^N \lambda_j^s \,\bigl(u_j^N(t)\bigr)^2.
\]
Multiplying each ODE by $\lambda_j^s\,u_j^N(t)$ and summing yields a Caputo ODE 
$\partial_t^{\alpha(\cdots)}E_N(t)\leq a\,E_N(t)+b$, similarly to Parts~A--C. Thus a 
\emph{fractional Grönwall} lemma gives a bound $E_N(t)\le M$ independent of $N$, implying 
\[
\|U_N\|_{L^\infty(0,T;H_0^s(\mathcal{M}))} \le \sqrt{M}.
\]
We omit the repetition of detailed steps (see Part~A’s product rule and Lemma~\ref{lem:fr_product_rule_expande}), 
which hold verbatim on $\mathcal{M}$.

\subsubsection{Passing to the Limit}

By Banach–Alaoglu (or a fractional Aubin--Lions–Simon argument), we extract a subsequence 
$U_{N_k}\rightharpoonup U$ in $H_0^s(\mathcal{M})$. We also handle the Caputo derivative 
$\partial_t^{\alpha(\cdots)}U_{N_k}$ via Volterra compactness (see Part~B for variable-order 
or Part~C for amplitude-triggered transitions). Identifying the limit in the PDE yields
\[
\partial_t^{\alpha(\cdots)}U
+ \mathcal{L}_{\mathbf{z}}^{s(\cdots)}\,U
=
\mathcal{F}\bigl(U\bigr),
\quad
U(0)=U_0,
\quad
U\big|_{\partial\mathcal{M}}=0 \text{ (or decay)},
\]
proving existence. Uniqueness under Lipschitz (or monotone) $\mathcal{F}$ follows from 
the same fractional Grönwall/difference-of-solutions argument as in Parts~A--C.

\begin{remark}[Switching, Memory Reinitialization]
If $\alpha(\cdots)$ or $s(\cdots)$ changes with amplitude threshold, we subdivide the 
time interval at threshold crossing times, just as done in Part~C. On each sub-interval, 
the PDE has a fixed or smoothly variable exponent, ensuring the Galerkin approach is 
valid piecewise in time. 
\end{remark}

Thus, the **Galerkin approximation** plus **fractional Grönwall** approach adapts 
straightforwardly to \(\mathcal{M}\), yielding existence, uniqueness, and energy bounds 
in $H_0^s(\mathcal{M})$ for the extended PDE. 
%=========================================================
%=========================================================
\subsection{Passing to the Limit and Uniqueness (Part D)}
\label{sec:limit_D}
%=========================================================

After constructing the Galerkin approximations $U_N(t,\mathbf{z})$ in $H_0^s(\mathcal{M})$ 
and establishing uniform energy bounds (Section~\ref{sec:galerkin_D}), we now show how to 
\emph{pass to the limit} $N\to\infty$ and ensure \emph{uniqueness} in the extended manifold 
setting. The essential arguments mirror those in Parts~A--C (single-domain); here, we 
highlight only the key steps for $\mathbf{z}\in \mathcal{M}$.

\subsubsection{Weak Convergence and Volterra Compactness}

\paragraph{Weak Convergence in Space.}
From the uniform bound 
\[
\|U_N\|_{L^\infty(0,T;H_0^s(\mathcal{M}))} \;\le\; C,
\]
we apply Banach–Alaoglu to extract a subsequence $U_{N_k}$ converging \emph{weakly} 
in $H_0^s(\mathcal{M})$ for each fixed $t\in[0,T]$. That is,
\[
U_{N_k}(t,\cdot) \;\rightharpoonup\; U(t,\cdot)
\quad
\text{in }H_0^s(\mathcal{M}).
\]
See Part~A for the analogous single-domain statement.

\paragraph{Fractional Time Derivative \texorpdfstring{\(\partial_t^\alpha\)}{} via Volterra Methods.}
As discussed in Part~B (variable-order) or Part~C (amplitude triggers), the Caputo 
derivative $\partial_t^{\alpha(\cdots)} U$ is a \emph{Volterra‐type} convolution in time. 
Standard results (cf.\ \cite[Chs.~3--4]{Diethelm2010} and references therein) imply that 
**boundedness** of $\mathbf{u}_N(t)$ in a finite‐dimensional ODE sense can yield 
\emph{time compactness} for $U_N$ in $H_0^s(\mathcal{M})$. Thus, $U_{N_k}$ becomes 
\emph{equicontinuous} in $t$, letting us pass to a limit $U$ in an appropriate sense 
(either strong in $L^p$ or weak$^*$ in $L^\infty$).

\paragraph{Limit Identification.}
Under these conditions:
\begin{itemize}
    \item $\mathcal{L}_{\mathbf{z}}^{s(\cdots)}U_{N_k} \rightharpoonup \mathcal{L}_{\mathbf{z}}^{s(\cdots)}U$ 
          weakly in $H_0^s(\mathcal{M})$, by continuity of the fractional Laplacian 
          (just as in single-domain, now for $\mathbf{z}\in \mathcal{M}$).
    \item $\mathcal{F}(U_{N_k}) \to \mathcal{F}(U)$ in a suitable sense 
          (pointwise or $L^p$) if $\mathcal{F}$ is Lipschitz or monotone. 
    \item $\partial_t^{\alpha(\cdots)} U_{N_k} \rightharpoonup \partial_t^{\alpha(\cdots)} U$ 
          in the distributional sense, thanks to Volterra stability arguments (Parts~B--C).
\end{itemize}
Hence the limit $U(t,\mathbf{z})$ solves
\[
\partial_t^{\alpha(\cdots)} U
+ 
\mathcal{L}_{\mathbf{z}}^{s(\cdots)} U
=
\mathcal{F}\bigl(t,\mathbf{z},U,\nabla_{\mathbf{z}}U\bigr),
\quad
U(0,\mathbf{z})=U_0(\mathbf{z}),
\quad
U|_{\partial\mathcal{M}}=0 \text{ (or decay)},
\]
in a weak sense, concluding \emph{existence} in $L^\infty(0,T;H_0^s(\mathcal{M}))$.

\subsubsection{Uniqueness Under Lipschitz or Monotonic \texorpdfstring{\(\mathcal{F}\)}{}}

\paragraph{Difference‐of‐Solutions Argument.}
If $U^{(1)},U^{(2)}$ are two solutions with the same initial data, we let 
$D=U^{(1)}-U^{(2)}$ and subtract the PDEs. The resulting difference PDE has 
\[
\partial_t^{\alpha(\cdots)} D
+ 
\mathcal{L}_{\mathbf{z}}^{s(\cdots)} D
=
\mathcal{F}(U^{(1)}) - \mathcal{F}(U^{(2)}).
\]
(See Part~A for details in the single-domain version.) If $\mathcal{F}$ is globally 
Lipschitz or monotone in $U$, one obtains a fractional Grönwall‐type estimate 
$\partial_t^{\alpha(\cdots)} \|D\|^2 \le \dots$ forcing $D\equiv 0$. Thus 
$U^{(1)}=U^{(2)}$ uniquely, as in Part~A or Part~C. 

\paragraph{Amplitude Switching.}
If $\alpha(\cdots)$ or $s(\cdots)$ switch upon amplitude threshold crossing, the argument 
applies \emph{piecewise in time} with continuity at each threshold time (Part~C). Provided 
$\mathcal{F}_{\mathrm{high}}$ remains Lipschitz or monotone in $U$, uniqueness is still 
assured.

\paragraph{Conclusion.}
Hence \emph{existence and uniqueness} extend naturally to the extended manifold PDE, 
yielding a \textbf{well-posed solution} 
\[
U \;\in\; L^\infty\bigl(0,T;H_0^s(\mathcal{M})\bigr).
\]
If blowup occurs (e.g.\ unbounded growth in the high-amplitude regime), one obtains only a 
\emph{maximal time} solution. Otherwise, $U$ persists on $[0,T]$.

\begin{theorem}[Well-Posedness in the Extended Domain \(\mathcal{M}\)]
\label{thm:ExtendedWellPosed}
Assume:
\begin{itemize}
  \item $U_0\in H_0^s(\mathcal{M})$, boundary/decay conditions on $\partial\mathcal{M}$,
  \item $\alpha(\cdots), s(\cdots)$ piecewise continuous and possibly switching via amplitude triggers,
  \item $\mathcal{F}$ is (globally) Lipschitz or monotone,
  \item The memory reinitialization rules (if any) are handled piecewise, per Part~C.
\end{itemize}
Then the Galerkin + fractional Grönwall + compactness argument 
yields a unique weak solution 
\[
U\in L^\infty\bigl(0,T;H_0^s(\mathcal{M})\bigr).
\]
\emph{If blowup is possible in the high-amplitude regime}, solutions exist only up to 
a maximal time $T_{\mathrm{max}}\le T$, beyond which classical or weak solutions cannot 
be continued.

In particular, 
\[
U \,\in\, L^\infty\bigl(0,T;H_0^s(\mathcal{M})\bigr)
\]
solves 
\(\partial_t^{\alpha(\cdot)} U + (-\Delta_{\mathbf{z}})^s U = \mathcal{F}\bigl(U\bigr)\) 
\emph{in the weak sense} on \([0,T]\times \mathcal{M}\).  
Boundary or decay conditions on \(\partial\mathcal{M}\) (and memory reinitialization 
if relevant) are enforced piecewise in time. 
Uniqueness follows by comparing two solutions and applying fractional Grönwall.

\end{theorem}

%=========================================================
\subsection{Harmonic Influence and Cross-Dimensional Probability Coupling}
\label{sec:harmonic-lowerdim}
%=========================================================

In \textbf{Part~D}, we have placed the PDE in an extended manifold 
\(\mathcal{M}=\Omega\times\mathcal{A}\). A key advantage of this setup is that 
\emph{wave-like interference} or \emph{harmonic forcing} in certain dimensions can 
affect \textbf{amplitude thresholds} or \textbf{probability distributions} in other dimensions, 
\emph{bi-directionally}. Below, we outline how lower-dimensional wave patterns can serve 
as attractors for higher-dimensional variables (e.g.\ “gravity-like” stabilization), and 
conversely how higher-dimensional amplitude expansions can trigger threshold events in 
the lower-dimensional domain.

\subsubsection{Bi-Directional Harmonic Forcing Across Dimensions}
\label{subsubsec:bi_directional_influence}

\paragraph{(1) Lower-Dimensional Wave Attractors for Higher-Dimensional Amplitude.}
\label{subsec:lower_dim_prob}
Consider the case where $\Omega$ is a lower-dimensional physical space (e.g.\ $n=1,2,3$) 
with classical wave forcing $\sin(\kappa \cdot x - \omega t)$ embedded. 
Although $\mathcal{A}$ might be higher-dimensional ($m\ge1$), wave solutions in the 
\emph{lower-dimensional} $x$-coordinates can effectively create \textbf{potential wells} 
or attractors in amplitude space $a\in\mathcal{A}$. 
For instance:
\begin{itemize}
  \item \emph{Gravity-like Stabilization:} If the PDE includes coupling terms 
        $\Phi(x,a)\,U(t,x,a)$ that \emph{depend} on wave modes in $x$, these modes can 
        shape how amplitude lumps in $a$ form stable attractors or orbits 
        (cf.\ “gravity wells” in a lower-dimensional subspace).
  \item \emph{Valence Shell Analogy:} In a quantum-like system, an electron might 
        “jump shells” if the amplitude coordinate $a$ crosses a critical threshold.  
        Yet wave interference in $x$ sets the conditions under which amplitude 
        can surmount that threshold. 
\end{itemize}
Hence, “low-dimensional wave patterns” can direct “high-dimensional amplitude transitions” 
in a single PDE, by controlling where and when amplitude lumps appear.

\paragraph{(2) Higher-Dimensional Wave Probability Affecting Lower Dimensions.}
Conversely, if $a\in\mathcal{A}$ supports wave-like or probabilistic expansions 
(e.g.\ “excited states” in amplitude coordinates), then these higher-dimensional modes 
can feed back into the \emph{physical} domain $x\in\Omega$. Examples:
\begin{itemize}
  \item \emph{Amplitude-Induced Blowups in $x$:} 
        A wave resonance in the amplitude dimension might push certain regions of 
        $(x,a)$ to exceed $A_{\mathrm{crit}}$, effectively triggering blowups or 
        threshold switching in $x$-space. 
  \item \emph{Quantum-Like Jumps in Lower Dimension:} 
        Probability lumps in $\mathcal{A}$ might localize near discrete amplitude levels, 
        altering how the PDE solution in $x$ evolves (e.g.\ a valence electron transition 
        that changes the effective potential in $x$).
\end{itemize}
This \textbf{cross-dimensional} interplay exemplifies how wave or probability distributions 
in one dimension shape threshold events in another.

\paragraph{Additional Remarks on Harmonic Forcing}\label{sec:harmonic_forcing}
It is important to note that the oscillatory nature of terms such as 
\(\sin(\kappa \cdot x - \omega t)\) and \(\cos(\kappa \cdot x - \omega t)\) does not affect the Lipschitz continuity of the overall forcing term. Since sine and cosine are smooth, bounded functions, any forcing tSRFT of the form \(\sin(\kappa \cdot x - \omega t)\,U\) remains globally (or piecewise) Lipschitz in \(U\). Consequently, when these terms are incorporated into the energy estimates, they contribute only bounded multiplicative factors. This boundedness ensures that the standard fractional Grönwall inequality applies without additional complications. Thus, harmonic forcing not only preserves well-posedness but also enriches the dynamics by facilitating cross-dimensional coupling while preserving existence and uniqueness of solutions.

\subsubsection{Harmonic Terms and Well-Posedness: Basic Observations}

\paragraph{Harmonic Forcing is Typically Lipschitz in $U$.}
As in Section~\ref{sec:harmonic_forcing}, adding factors such as 
\(\sin(\kappa \cdot x - \omega t)\,U\) or 
\(\cos(\kappa \cdot x - \omega t)\,U\)
does not break the well-posedness arguments, since these remain globally (or piecewise) 
Lipschitz in $U$. The same \emph{energy estimate} and \emph{fractional Grönwall} logic 
apply, so we see no obstruction to existence or uniqueness.

\paragraph{Interference-Driven Threshold Crossings.}
By mixing wave forcing with amplitude-threshold logic, one obtains 
\emph{localized lumps} precisely where constructive interference pushes 
$|U|>A_{\mathrm{crit}}$. Meanwhile, destructive interference prevents threshold 
crossings in other regions. 
In a multi-dimensional manifold $\mathcal{M}=\Omega\times\mathcal{A}$, these lumps can 
be \emph{bi-directionally} controlled: wave resonances in $\Omega$ can cause amplitude 
transitions in $\mathcal{A}$ (and vice versa), all captured in \emph{one PDE}.

\subsubsection{Interpretation via Probability Projection}
\label{subsubsec:prob_projection}

Just as in Section~\ref{subsec:lower_dim_prob}, one can integrate out the amplitude 
coordinates $a\in\mathcal{A}$ to obtain a “marginal” distribution 
\(\rho(t,x)=\int_{\mathcal{A}}|U|^2\,da\). This bridging from a higher-dimensional 
wave-amplitude PDE to a \emph{lower-dimensional} probability-like measure in $x$ 
permits wave-particle interpretations:
\begin{itemize}
  \item \emph{Stable Attractors or Discrete Jumps}: 
        Probability lumps in $a$ appear at certain amplitude levels, reflecting 
        stable solutions or “shells.” Meanwhile, $\rho(t,x)$ reveals how these lumps 
        manifest as detection events in $x$. 
  \item \emph{Born-Rule-like Behavior}: 
        Over many realizations, lumps appear at $x^*$ with frequency proportional to 
        $\rho(t,x^*)$. 
  \item \emph{Dimensional Coupling}: 
        Wave structures in $x$ can persist or vanish depending on $a$-threshold triggers, 
        while lumps in $a$ reflect constructive interference from $x$-waves, and so on.
\end{itemize}

Hence, \textbf{the extended SRFT PDE} on $\Omega\times\mathcal{A}$ naturally accommodates 
\emph{cross-dimensional wave interference, amplitude thresholds, blowups, and 
probability-like projection}---paving the way for models that unify classical wave 
phenomena and discrete quantum-like events in a single PDE framework.

%=========================================================
%=========================================================
\subsection{Threshold Switching, Memory, and Blowups (Part D)}
\label{sec:thresholds_d}
%=========================================================

In \textbf{Part~C}, we established how \emph{amplitude‐triggered} fractional exponents 
and blowup scenarios work in a \emph{single-domain} \(\Omega\). 
Here, we emphasize that the \textbf{same piecewise} and \textbf{memory reinitialization} 
logic extends to the \emph{higher‐dimensional} setting \(\mathcal{M}=\Omega\times\mathcal{A}\), 
\emph{locally} in \(\mathbf{z}\)‐space. That is, once \(\lvert U(t,\mathbf{z})\rvert\) 
crosses a threshold \(A_{\mathrm{crit}}\) at certain points \(\mathbf{z}_*\), the PDE 
switches from $(\alpha_{\mathrm{low}},s_{\mathrm{low}},F_{\mathrm{low}})$ to 
$(\alpha_{\mathrm{high}},s_{\mathrm{high}},F_{\mathrm{high}})$ \emph{in subregions} of 
\(\mathbf{z}\). Below, we briefly outline the key steps.

\subsubsection{Local PDE Switching in \texorpdfstring{\(\mathbf{z}\)‐Space}{z-space}}
\label{subsubsec:localPDEswitch_zspace}

Define
\[
\alpha(\mathbf{z},U) 
=
\begin{cases}
\alpha_{\mathrm{low}}(\mathbf{z}), & \lvert U\rvert \le A_{\mathrm{crit}},\\[4pt]
\alpha_{\mathrm{high}}(\mathbf{z}), & \lvert U\rvert > A_{\mathrm{crit}},
\end{cases}
\quad
s(\mathbf{z},U)
=
\begin{cases}
s_{\mathrm{low}}(\mathbf{z}), & \lvert U\rvert \le A_{\mathrm{crit}},\\[4pt]
s_{\mathrm{high}}(\mathbf{z}), & \lvert U\rvert > A_{\mathrm{crit}}.
\end{cases}
\]
Plus, if the forcing $\mathcal{F}(\mathbf{z},U)$ also changes above threshold, we split 
$\mathcal{F}_{\mathrm{low}}$ vs.\ $\mathcal{F}_{\mathrm{high}}$. Unlike the single-domain 
case, \emph{only certain subsets} of $\mathbf{z}\in\mathcal{M}$ may exceed the threshold, 
so the PDE can be in different fractional regimes in different \emph{patches} of 
$(x,a)$‐space. This naturally accommodates \emph{finite-time blowups} or \emph{localized 
lumps} restricted to certain subregions of $\mathcal{M}$.

\subsubsection{Memory Reinitialization vs.\ Global Memory}
\label{subsubsec:memory_extended}

As before (Part~C), we must decide if the Caputo kernel resets at $t_b$ (the first time 
$\lvert U\rvert$ hits $A_{\mathrm{crit}}$) or continues from $0$. Both:

\begin{itemize}
\item \textbf{Global Memory}:
  The integral is always $\int_0^t \!\!(t-\tau)^{-\alpha(\cdots)}\,U'(\tau)\,d\tau$, 
  so past states remain relevant even after threshold crossing.
\item \textbf{Local/Reset Memory}:
  The integral restarts at $t_b$ or at each crossing time. This “forgets” prior amplitude 
  regimes, akin to partial memory reinitialization.
\end{itemize}

In $\mathcal{M}$, each threshold crossing can happen \emph{simultaneously} in multiple 
patches of $(x,a)$, or asynchronously in different pockets. The piecewise PDE 
(\S\ref{sec:galerkin_D}--\ref{sec:limit_D}) can be re-solved on each sub-interval of time 
containing a new crossing event, just as in Part~C, but now tracking $\mathbf{z}$‐local 
transitions.

\subsubsection{Blowup, Lumps, or Saturation}
\label{subsubsec:blowup_extended}

If $\mathcal{F}_{\mathrm{high}}$ is unbounded (e.g.\ superlinear in $U$), 
finite‐time blowups may occur in localized pockets of $(x,a)\in\mathcal{M}$. 
Alternatively, saturating $\mathcal{F}_{\mathrm{high}}$ yields \emph{stable lumps} 
where amplitude saturates at $|U|\approx A_{\mathrm{crit}}$.  
In either scenario:
\begin{itemize}
  \item \emph{Finite-Time Blowup}:
    The classical/weak solution terminates at a maximal time $T_{\mathrm{max}}<T$. 
    Beyond that, no continued solution is possible unless adopting measure-valued or 
    distributional frameworks.
  \item \emph{Saturation or Clamping}:
    Amplitude does not diverge but “freezes” at $A_{\mathrm{crit}}$, forming lumps in 
    certain $\mathbf{z}$‐regions. 
\end{itemize}

\subsubsection{Relation to Harmonic Interference \& Probability Dimensions}

Because wave forcing in $\Omega$ can \emph{amplify} $U$ at constructive interference 
fringes, those fringes often cross $A_{\mathrm{crit}}$ first. Similarly, amplitude 
states in $\mathcal{A}$ can “feed back” by altering the PDE coefficients in $x\in\Omega$. 
Hence, lumps or blowups typically emerge \emph{exactly} where wave amplitude is largest, 
tying wave interference to amplitude thresholds in a single PDE. 
(See \S\ref{sec:harmonic-lowerdim} for a deeper discussion.)

\begin{remark}[Multiple Thresholds or “Multi-Regime” Transitions]
If there are multiple critical levels $0<A_{\mathrm{crit}}^{(1)}<A_{\mathrm{crit}}^{(2)}<\cdots$, 
the PDE can switch among several $(\alpha,s,\mathcal{F})$ “bands,” 
leading to fractal or layered lumps in $(x,a)$. 
The piecewise approach extends naturally, subdividing time intervals at each threshold 
crossing event. Uniqueness or blowup must be re-examined in each regime.
\end{remark}

Hence, **amplitude-triggered threshold logic** merges naturally with \emph{wave interference} 
and higher-dimensional expansions in $\mathcal{M}=\Omega\times\mathcal{A}$. Through 
**localized PDE switching**, memory choices (global vs.\ partial), and blowup or lump 
formation, Part~D extends the single-domain blowup framework (Part~C) to a full 
\textbf{multi-dimensional amplitude-and-wave} setting.

\subsection*{Conclusion for Part D}
\addcontentsline{toc}{section}{Conclusion for Part D}
\label{sec:conclusion_partD}

In this \textbf{Part~D}, we extended the single-domain fractional PDE theory (Parts~A--C) to a 
\emph{higher-dimensional manifold}
\[
\mathcal{M} \;=\; \Omega\times\mathcal{A},
\]
where \(\Omega\subset \mathbb{R}^n\) represents physical space and \(\mathcal{A}\subset \mathbb{R}^m\) 
encodes amplitude or probability coordinates. We demonstrated how the key ingredients of 
Galerkin approximation, fractional Gr\"onwall estimates, and compactness (Aubin--Lions--Simon 
type) carry over from the single-domain setting. In particular:

\begin{itemize}
    \item \textbf{Spectral Fractional Laplacian on \(\mathcal{M}\).} We defined $(-\Delta_{\mathbf{z}})^s$ in the product domain 
          \(\mathbf{z}=(x,a)\in\mathcal{M}\), ensuring that Dirichlet or decay conditions 
          in both physical and amplitude directions yield a valid fractional operator 
          (Section~\ref{subsec:extended_manifold_setup}).
    \item \textbf{Well-Posedness via Galerkin \& Fractional Gr\"onwall.} We constructed 
          approximate solutions $U_N$ by truncating an $L^2(\mathcal{M})$ eigenbasis, 
          leading to a system of Caputo ODEs in time. A fractional energy estimate then 
          implied uniform bounds in $H_0^s(\mathcal{M})$, and passing to the limit showed 
          existence of a weak solution 
          (Sections~\ref{sec:galerkin_D}--\ref{sec:limit_D}). Uniqueness followed by 
          comparing two solutions and applying the same fractional Gr\"onwall argument 
          under Lipschitz or monotonic forcing.
    \item \textbf{Amplitude-Triggered Thresholds \& Blowups.} As in Part~C, amplitude-based 
          switching (e.g., $\lvert U\rvert>A_{\mathrm{crit}}$) still applies in 
          \(\mathcal{M}\), but now \emph{locally} in $(x,a)$-space. We showed how memory 
          reinitialization vs.\ global memory, blowups or saturations, and wave forcing 
          interact in a single PDE framework (Section~\ref{sec:thresholds_d}).
    \item \textbf{Harmonic Influence \& Probabilistic Interpretations.} By allowing harmonic 
          or wave forcing in $\Omega$ and amplitude-based lumps in $\mathcal{A}$, we 
          captured wave-particle-like phenomena within one PDE: interference patterns can 
          trigger amplitude blowups, or amplitude lumps can re-shape low-dimensional wave 
          attractors. A probability-like distribution emerges by integrating out $\mathcal{A}$, 
          thus connecting wave interference to detection “clicks” (\S\ref{sec:harmonic-lowerdim}).
\end{itemize}

Overall, \textbf{Part~D} shows that \emph{all the fundamental ideas} from the SRFT approach---%
Galerkin approximations, fractional memory, threshold switching, wave forcing, and 
probabilistic interpretations---extend naturally from the single-domain PDE to 
\(\mathcal{M}\!=\!\Omega\times\mathcal{A}\). This unification enables a powerful, 
\emph{one-PDE} view of multi-dimensional wave-amplitude dynamics, supporting amplitude-driven 
threshold events, local or global memory, and blowup/saturation phenomena in a single, 
deterministic fractional PDE framework.

\medskip

\noindent
In future investigations, one might explore:
\begin{itemize}
    \item \textbf{Multiple Amplitude Thresholds or Fractal Transitions}: 
          Allowing multiple critical amplitude levels can yield intricate blowup patterns 
          or fractal structures in $(x,a)$-space.
    \item \textbf{Noise or Chaos in Higher Dimensions}: 
          Injecting noise terms may produce irreproducible single events with frequency 
          tied to $|U|^2$, reinforcing the quantum-like Born-rule outcome in a purely 
          classical PDE context.
    \item \textbf{Complex Geometries or Multi-Particle Coordinates}: 
          If $\Omega$ represents multi-particle domains or $\mathcal{A}$ is multi-leveled, 
          the dimension of $\mathcal{M}$ can grow significantly, raising questions about 
          numerical feasibility and large-scale parallel computing.
    \item \textbf{Variable-Order in Both \(t\) \emph{and} \(\mathbf{z}\)}: 
          Time- and space-dependent fractional exponents can further enrich the memory or 
          amplitude-triggered logic, requiring advanced stability arguments.
\end{itemize}

Thus, \textbf{Part~D} completes the extended SRFT picture, forging a link between traditional 
wave interference and discrete amplitude thresholds in higher-dimensional manifolds, all 
under a single well-posed fractional PDE paradigm.

%=========================================================
\section*{Conclusion and Future Directions}
\addcontentsline{toc}{section}{Conclusion and Future Directions}
\label{sec:conclusion_monograph}
%=========================================================

\paragraph{Monograph Summary.}
In this monograph, we developed a unified \emph{fractional PDE} framework for modeling 
\textbf{nonlocal memory}, \textbf{amplitude-triggered threshold events}, and 
\textbf{multi-dimensional wave interference}, culminating in a single PDE that can switch 
fractional exponents upon crossing critical amplitude levels. Our approach was divided into 
four main parts:

\begin{enumerate}
\item \textbf{Part~A: Fixed-Order Fractional PDE.} We introduced the classical Caputo 
      derivative \(\partial_t^\alpha\) with $\alpha\in(0,1)$ constant and a spectral 
      fractional Laplacian $(-\Delta)^s$ in a single bounded domain \(\Omega\). 
      Through a Galerkin-plus-fractional-Grönwall approach, we established 
      \emph{existence, uniqueness, and energy bounds} in \(H_0^s(\Omega)\).

\item \textbf{Part~B: Variable-Order Extensions.} We allowed $\alpha(t)$ (and potentially 
      $s(t)$) to vary over time, retaining well-posedness by subdividing time intervals 
      and applying a \emph{Volterra stability} argument. The core techniques—Galerkin 
      approximation, fractional product rules, Grönwall inequalities—remained fundamentally 
      the same, only applied piecewise for each sub-interval where $\alpha(t)$ was constant 
      or smoothly varying.

\item \textbf{Part~C: SRFT Framework in Single Domain (Amplitude-Triggered Blowups).} 
      We incorporated \emph{amplitude thresholds}: the PDE switched fractional orders (or 
      forcing) whenever $|A(t,x)|$ surpassed a critical amplitude $A_{\mathrm{crit}}$. 
      This unified \emph{finite-time blowups, lumps, or saturation} with fractional memory 
      reinitialization vs.\ global memory. Local PDE switching in time/space made it 
      possible to handle multi-regime transitions and piecewise well-posedness.

\item \textbf{Part~D: Extended Manifold \(\mathcal{M}=\Omega\times \mathcal{A}\).} 
      Finally, we moved to a \emph{higher-dimensional} domain capturing both \textbf{wave 
      interference} in $x\in\Omega$ and \textbf{amplitude-based thresholds} in $a\in\mathcal{A}$. 
      By embedding wave forcing and fractional memory in a single PDE on 
      \(\Omega\times \mathcal{A}\), we reproduced classical interference patterns 
      while amplitude triggers led to local blowups or saturations. A \emph{probability-like} 
      distribution arose by integrating out amplitude coordinates, linking wave interference 
      to discrete detection events.
\end{enumerate}

Across all parts, the \emph{Galerkin approach} plus \emph{fractional Grönwall estimates} 
provided uniform bounds and led to a limit PDE solution via compactness arguments. 
Uniqueness followed by comparing two solutions and invoking Lipschitz or monotone forcing 
conditions.

\paragraph{Key Insights and Achievements.}
\begin{itemize}
\item \textbf{Fractional Memory and Threshold Switching:} We showed how partial or global 
      reinitialization of the Caputo kernel can handle amplitude-triggered exponents. 
      This permits local blowups, lumps, or saturations within a mathematically rigorous 
      PDE framework.
\item \textbf{Wave Interference and Amplitude Lumps in One PDE:} 
      Embedding amplitude coordinates into a higher-dimensional manifold $\Omega\times \mathcal{A}$ 
      unifies wave-like forcing with amplitude-based blowups or transitions. Over repeated 
      runs, lumps emerge with frequency proportional to $|U|^2$, reminiscent of quantum 
      detection probabilities.
\item \textbf{Piecewise Well-Posedness for Multi-Regime PDEs:} 
      Subdividing time intervals at threshold crossings and re-applying Galerkin ensures 
      continuity of solutions. This structure generalizes readily to multi-level threshold 
      phenomena or spatially varying fractional orders.
\end{itemize}

\paragraph{Future Directions.}
Though the monograph has addressed a broad class of fractional PDEs, several \emph{open 
problems} and \emph{extensions} remain:

\begin{enumerate}
\item \textbf{Multiple Thresholds or Fractal Lumps.} 
  Allowing multiple critical amplitudes $A_{\mathrm{crit}}^{(1)}<A_{\mathrm{crit}}^{(2)}<\dots$ 
  can yield layered or fractal blowups, requiring a careful piecewise scheme for each 
  threshold crossing. Uniqueness may become subtle if different branches of exponents 
  overlap.

\item \textbf{Noisy or Chaotic Seeds.} 
  Introducing small noise or deterministic chaos (e.g.\ random initial phases) can break 
  symmetry and produce irreproducible events. One might interpret these events as 
  “clicks” distributed according to $|U|^2$. Detailed numerical analysis or measure-theoretic 
  PDE methods could clarify this link.

\item \textbf{Entangled Coordinates or Nonlocal Couplings.}
  In advanced physical models, amplitude or wave variables may be \emph{shared} across 
  multiple particles, hinting at multi-body PDE analogs of quantum entanglement. The 
  dimension of $\mathcal{M}$ grows quickly, testing the feasibility of direct Galerkin 
  expansions.

\item \textbf{Variable-Order Extensions in Multi-Dimensions.} 
  Many real systems exhibit memory parameters $\alpha(\mathbf{z},t)$ that shift with 
  amplitude or wave feedback. A \emph{variable} $s(\mathbf{z},t)$ might also reflect 
  changing spatial fractality. Handling these with piecewise fractional Grönwall arguments 
  is possible but more intricate.

\item \textbf{Efficient Numerical Schemes in Large \(\mathcal{M}\).}
  Even in the single-domain case, fractional PDEs are numerically challenging. In 
  $\mathcal{M}=\Omega\times\mathcal{A}$, the dimension grows, and threshold logic 
  necessitates adaptive time-stepping or domain decomposition. Developing robust, 
  parallelizable algorithms is crucial for practical applications (e.g.\ high-dimensional 
  wave-amplitude problems).
\end{enumerate}

\paragraph{Concluding Remarks.}
By integrating \textbf{fractional memory}, \textbf{wave interference}, and 
\textbf{amplitude-triggered transitions} into one deterministic PDE over an 
\emph{extended manifold}, we have provided a framework unifying continuous wave phenomena 
with discrete threshold events. This \emph{Self Referential Field Theory (SRFT)} perspective 
naturally accommodates local blowups, saturations, multi-level amplitude thresholds, and 
probability-like outcomes reminiscent of quantum detection. We hope these results stimulate 
further exploration of \emph{multi-dimensional fractional PDEs}, bridging classical 
wave equations and discrete amplitude logic in new and fascinating ways.

\section{Authors Note}
Although my formal training in advanced mathematics is limited (my most recent course was over three decades ago), I have in recent months employed AI tools—especially large language models—to accelerate my reading, literature research, and initial manuscript drafting. These tools helped me bridge knowledge gaps and integrate diverse sources more efficiently. Nevertheless, the conceptual framework and conclusions in this paper are solely my own, and I accept full responsibility for any errors.

I recognize that fractional PDEs and related fields have a rich, rigorous tradition underpinned by the work of many expert researchers. I welcome constructive feedback, corrections, and potential collaborations with those who bring deeper domain expertise. My hope is that this paper contributes to ongoing discussions and opens new avenues for inquiry in amplitude-triggered fractional PDEs.

\bigskip

%=========================================================
% REFERENCES
%=========================================================
\begin{thebibliography}{99}

\bibitem{Diethelm2010}
K.~Diethelm.
\newblock \emph{The Analysis of Fractional Differential Equations}.
\newblock Springer, 2010.

\bibitem{DiNezzaPalatucciValdinoci}
E.~Di~Nezza, G.~Palatucci, and E.~Valdinoci.
\newblock Hitchhiker's guide to the fractional {S}obolev spaces.
\newblock \emph{Bull. Sci. Math.}, 136(5):521--573, 2012.

\bibitem{Kilbas2006}
A.~A. Kilbas, H.~M. Srivastava, and J.~J. Trujillo.
\newblock \emph{Theory and Applications of Fractional Differential Equations}.
\newblock Elsevier, 2006.

\bibitem{LinXu}
Y.~Lin and C.~Xu.
\newblock Finite difference/spectral approximations for the time-fractional diffusion equation.
\newblock \emph{J. Comput. Phys.}, 225(2):1533--1552, 2007.

\bibitem{Sun2019}
H.~Sun, A.~Chang, Y.~Zhang, and W.~Chen.
\newblock A review on variable-order fractional differential equations: Mathematical foundations, physical models, numerical methods and applications.
\newblock \emph{Fractional Calculus and Applied Analysis}, 22(1):27--59, 2019.

\bibitem{Magin2006}
R.~L. Magin.
\newblock \emph{Fractional Calculus in Bioengineering}.
\newblock Begell House, 2006.

\bibitem{Simon1987}
J.~Simon.
\newblock Compact sets in the space {$L^p(0,T;B)$}.
\newblock \emph{Ann. Mat. Pura Appl.}, 146(1):65--96, 1987.

\bibitem{Ye2007}
H.~Ye, J.~Gao, and Y.~Ding.
\newblock A generalized {G}ronwall inequality and its application to a fractional differential equation.
\newblock \emph{J. Math. Anal. Appl.}, 328(2):1075--1081, 2007.

\bibitem{Zacher2005}
R.~Zacher.
\newblock Weak solutions of evolutionary integro-differential equations in Hilbert spaces.
\newblock \emph{Funkcialaj Ekvacioj}, 48(1):1--27, 2005.


\end{thebibliography}

\end{document}
